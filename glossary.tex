\section{Appendix: Glossary}
\label{appendix:glossary}

\begin{description}

\item[\textbf{ACID}] stands for atomicity, consistency, isolation, and durability. "Atomicity" guarantees that any transaction completes, and if not, prevents it. "Consistency" ensures the stability of the database before and after any transaction. "Isolation" guarantees that parallel transactions occur independently. Finally, "Durability" maintains the state of the result of the transaction immutable \cite{jatana2012survey}.

\item[\textbf{BASE}] stands for basically available, soft state, and eventual consistency. "Basically Available" ensures high availability of a database, even during system failures, through sharding or partitioning, which distributes and potentially replicates data across multiple servers. ”Soft State” facilitates transaction processing despite system failures or disturbances, allowing for temporary inconsistencies. While eventual consistency is guaranteed, consistency control is shifted to the application or business logic layer, due to the absence of ACID. Finally, ”Eventually consistent” weakens the constraints of consistency at the end of every transaction and guarantees that the consistent state will be granted to data stores at a later stage \cite{mapanga2013database}.

\item[\textbf{CAP}] 
(Not explicitly defined in the text)

\item[\textbf{DBMS}] is a software application that enables users to define, manipulate, and manage databases. It provides a structured environment for efficiently storing, organizing, retrieving, and managing large volumes of data. This includes tasks from the initial database creation and schema definition to ongoing operations such as data entry, modification, deletion, querying, and report generation, ensuring data integrity, security, and accessibility \cite{rawat2021mysql}.

\item[\textbf{DTO}] encapsulates a set of values, enabling remote clients to retrieve the entire set in a single request. DTOs are stateless, possessing no business logic or data retrieval capabilities, and are independent of business objects, facilitating their reuse across various contexts and contract calls \cite{lee2018bridgify, pantaleev2007identifying, prakharenka2019development}.

\item[\textbf{IANA}] is responsible for coordinating some of the key elements that keep the Internet running smoothly. Whilst the Internet is renowned for being a worldwide network free from central coordination, there is a technical need for some key parts of the Internet to be globally coordinated, and this coordination role is undertaken by us.
Specifically, IANA allocates and maintains unique codes and numbering systems that are used in the technical standards (“protocols”) that drive the Internet. \cite{iana_about}.

\item[\textbf{JVM}]  is a crucial component of the Java platform, providing hardware and OS independence, compact compiled code, and protection against malicious programs. It is rogramming languages. The JVM is multithreaded, performs garbage collection, and generates events useful for profiling \cite{lindholm2014java, viswanathan2000java}.

\item[\textbf{JS}] is a flexible and widely adopted scripting language crucial for developing interactive Web 2.0 applications. It enables offloading core functionality to the client-side browser and dynamically manipulating the Document Object Model create seamless state transitions. While its flexibility offers significant advantages, it also presents challenges in writing and maintaining robust code. As an object-based scripting language, JavaScript is natively interpreted by most common web browsers \cite{fard2013jsnose, stothard2000sequence}.

\item[\textbf{MVC}] is a software design pattern that separates an application into three interconnected parts: the Model (data and business logic), the View (user interface), and the Controller (handles user input and updates the Model 1  and View). This separation enhances code organization, reusability, flexibility, and simplifies development and maintenance of web applications by providing a clear understanding of each module. The primary goal of MVC is to isolate the business logic and application data from its presentation to the user \cite{jivani2013over, singh2016comparative, thakur2019role, thakur2019study}.

\item[\textbf{RFC}] describes the Internet's technical foundations, such as addressing, routing, and transport technologies. RFCs also specify protocols like TLS 1.3, QUIC, and WebRTC that are used to deliver services used by billions of people every day, such as real-time collaboration, email, and the domain name system \cite{ietf_rfc_website}.

\item[\textbf{SDLC}] it is a process of building or maintaining software systems. Typically, it includes various phases from preliminary development analysis to post-development software testing and evaluation. It also consists of the models and methodologies that development teams use to develop the software systems, which the methodologies form the framework for planning and controlling the entire development process \cite{leau2012software}.

\item[\textbf{STLC}] 
    (Not explicitly defined in the text)

\item[\textbf{SQL}] is the standard language for retrieving and manipulating relational data.  It is a high-level non-procedural data language which has received wide recognition in relational databases \cite{blacher2022machine, kim1982optimizing}. 

\item[\textbf{UI}] serves as the crucial communication bridge between the system and the user. Effective user interface design encompasses all user-visible aspects of the system. Consequently, its design is deeply intertwined with the overall architecture of the interactive system and must be integrated from the outset of the development process, rather than as an afterthought. A well-designed user interface yields significant improvements in training efficiency, task performance speed, reduced error rates, enhanced user satisfaction, and improved long-term retention of operational knowledge. Critically, the user interface should be designed using the users's own terminology and understanding of their tasks, rather than adhering to the technical jargon and conceptual framework of the programmers \cite{jacob2003}.

\item[\textbf{Kotlin}]  is a modern but already mature programming language designed to make developers happier. It is concise, safe, interoperable with Java and other languages, and provides many ways to reuse code between multiple platforms for productive programming \cite{kotlin_getting_started}.

\item[\textbf{Kotin Coroutines}]  is an instance of a suspendable computation. It is conceptually similar to a thread, in the sense that it takes a block of code to run that works concurrently with the rest of the code. However, a coroutine is not bound to any particular thread. It may suspend its execution in one thread and resume in another one. Coroutines can be thought of as light-weight threads, but there is a number of important differences that make their real-life usage very different from threads \cite{kotlin_coroutine}.

\item[\textbf{UML}] serves as a standard visual language for specifying, constructing, and documenting the artifacts of systems. UML employs various diagrams to represent system models graphically. This graphical notation lacks a direct textual equivalent; instead, the Object Constraint Language is utilized to express model constraints in a textual format. As a general-purpose modeling language, UML is compatible with all major object and component methods and can be applied across diverse application domains and implementation platforms \cite{garcia2004uml, he2006comparison}.

\item[\textbf{URI}] is a compact sequence of characters that identifies an abstract or physical resource. As consequence, it is used throughout HTTP as the means for identifying resources \cite{rfc3986, rfc9110}.

\item[\textbf{Spring Boot}] is used as an alternative to deploy software products in application servers. It is considered the \textit{de facto} standard for microservice development. The framework has a built-in server by which the process of implementing a REST application is significantly simplified \cite{gomez2020crudyleaf}.

\item[\textbf{HATEOS}]  
(Not explicitly defined in the text)

\item[\textbf{XML}] 
(Not explicitly defined in the text)

\end{description}