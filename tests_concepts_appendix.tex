\section{Appendix: Definitions of Automated Software Tests}
\label{appendix:tests_concepts_appendix}

\begin{longtable}{@{}p{3cm}p{6cm}p{5cm}@{}}
    \caption{Tests and Their Definitions} \label{tab:tests_concepts} \\
    \toprule
    \textbf{Testing Type} & \textbf{Description} & \textbf{Key Aspects} \\
    \midrule
    \endfirsthead
    
    \multicolumn{3}{@{}c@{}}{\textbf{\tablename\ \thetable\ -- Continued from previous page}} \\
    \toprule
    \textbf{Testing Type} & \textbf{Description} & \textbf{Key Aspects} \\
    \midrule
    \endhead
    
    \midrule
    \multicolumn{3}{@{}r@{}}{Continued on next page} \\
    \endfoot
    
    \bottomrule
    \endlastfoot
    
    Unit Testing & Tests the smallest testable parts (units/modules) of an application independently. & 
    \begin{itemize}[leftmargin=*,nosep,after=\vspace{-\baselineskip}Context]
        \item Focuses on individual modules
        \item Also called module testing
        \item Typically done by developers
        \item Example: Testing individual functions like Addition, Subtraction in a Calculator
        \item Known as white box testing (tests internal functionality)
        \item Aims to eliminate interface errors early
    \end{itemize} \\
    \midrule
    
    Integration Testing & Tests the interaction and interfaces between integrated units/modules. &  
    \begin{itemize}[leftmargin=*,nosep,after=\vspace{-\baselineskip}Context]
        \item Exposes faults in the interaction between units
        \item Performed after unit testing
        \item Approaches: Big-bang, Top-down, Bottom-up
        \item Focuses on ensuring units work together correctly
    \end{itemize} \\
    \midrule
    
    System Testing & Tests the complete, integrated system to verify it meets specified requirements. &  
    \begin{itemize}[leftmargin=*,nosep,after=\vspace{-\baselineskip}Context]
        \item Testing of the entire software and potentially hardware system
        \item Last test by developers before acceptance testing
        \item Includes types like usability, stress, and regression testing
        \item Requires a System Test Plan
        \item Typically black box testing (no knowledge of internal design)
        \item Detects defects within inter-assemblages and the system as a whole
    \end{itemize} \\
    \midrule
    
    Acceptance Testing & Tests the system for acceptability by end users or clients. &  
    \begin{itemize}[leftmargin=*,nosep,after=\vspace{-\baselineskip}Context]
        \item High-level testing on a fully integrated application
        \item Types: Alpha, Beta, User Acceptance Testing (UAT), Business Acceptance Testing
        \item Done by end users
        \item Determines whether to accept or reject the product based on acceptance criteria
    \end{itemize} \\
    \midrule
    
    Regression Testing & Re-tests previously tested parts of the software after modifications (bug fixes, changes). &  
    \begin{itemize}[leftmargin=*,nosep,after=\vspace{-\baselineskip}Context]
        \item Checks for new bugs introduced by changes
        \item Verifies that previously reported bugs are fixed
        \item often involves re-running failed test cases
        \item Can be expensive but necessary for ensuring the correctness and reliability of modified software
        \item Aims to increase confidence in the modified program
        \item Targets potential errors like data corruption, inappropriate control sequencing, resource contention, and performance deficiencies
    \end{itemize} \\
    \midrule
    
    Installation Testing & Verifies if the software has been installed with all necessary components. &  
    \begin{itemize}[leftmargin=*,nosep,after=\vspace{-\baselineskip}Context]
        \item Checks if the application can be downloaded and installed
        \item Verifies registration processes (e.g., with a new mobile number)
        \item Confirms receipt of necessary verification codes
    \end{itemize} \\
    \midrule
\end{longtable}
\centering 
\footnotesize Source: \cite{khan2011different, meenakshi2014software}