\chapter{Governo eletrônico e digital no Brasil}

\cite{tavares2022governo} ressalta que a \href{https://www.planalto.gov.br/ccivil_03/constituicao/constituicao.htm}{Constituição Federal de 1988} fixou a cidadania como fundamento da República, tendo a participação e o controle papéis essenciais ao bom funcionamento do Estado, da Democracia e da Administração Pública, a partir da concepção de cidadania e democracia participativa.

Além disso, \cite{tavares2022governo} argumenta que o controle social possui estreita ligação com as políticas públicas, pois, a partir do seu exercício, em todas as etapas do ciclo, desde a formulação até a avaliação, confere-se maior legitimidade e eficiência aos resultados dos objetivos, metas e diretrizes fixadas pelos planos, programas e ações dentro do conjunto de políticas públicas.

Adicionalmente, \cite{tavares2022governo} afirma que as políticas públicas são a forma como se resolve os problemas da sociedade e o controle social é a forma como o cidadão interage, fiscaliza e questiona as soluções definidas para esses problemas. 

Como consequência, \cite{Guimaraes2005} argumenta que o governo eletrônico foi visto como uma oportunidade de incrementar a participação da sociedade na gestão pública, especialmente quanto à formulação, ao acompanhamento e à avaliação das políticas públicas, visando ao incremento da cidadania e da democracia. 

\cite{rover2009introduccao} argumenta que a interação entre as novas tecnologias, a sociedade e o Poder Público emoldura um momento único do qual emergem, simultaneamente, desafios enormes e vantagens sociais incríveis. Neste contexto, o aparecimento do governo eletrônico é uma decorrência das velhas e novas demandas da sociedade.

Para \cite{rover2009introduccao}, governo eletrônico é uma infra-estrutura única de comunicação compartilhada por diferentes órgãos públicos a partir da qual a TIC é usada de forma intensiva para melhorar a gestão pública e o atendimento ao cidadão.

Adicionalmente, como é entendido por \cite{rover2009introduccao}, o objetivo do governo eletrônico é colocar o governo ao alcance de todos, ampliando a transparências das suas ações e incrementando a participação cidadã, almejando a universalização de serviços.

Diversos autores destacam o impacto positivo do governo eletrônico na sociedade. Suas conclusões estão presentes na tabela \ref{tab:autores_beneficios_egov}.

\begin{longtable}{@{}p{0.18\textwidth} p{0.78\textwidth}@{}}
	\caption{Revisão da literatura} \label{tab:autores_beneficios_egov} \\ % Legenda da tabela
	\toprule
	\textbf{Autor} & \textbf{Conclusão} \\
	\midrule
	\endfirsthead
	
	\multicolumn{2}{@{}l@{}}{\textbf{Tabela \thetable{} -- continuação da página anterior}} \\
	\toprule
	\textbf{Autor} & \textbf{Conclusão} \\
	\midrule
	\endhead 
	
	\midrule
	\multicolumn{2}{r@{}}{\textit{Continua na próxima página}} \\
	\endfoot
	
	\bottomrule
	\multicolumn{2}{r@{}}{\footnotesize{Fonte: elaboração própria.}} \\
	\endlastfoot
	
	\cite{martins2022digital} &
	\RaggedRight Suas estimativas de que um nível alto de governo eletrônico podem facilitar negócios pela diminuição do fardo das regulações em diversas áreas de negócio. \\
	\midrule
	\cite{kotenok2020government} &
	\RaggedRight Conclui que o impacto do governo eletrônico pode impulsionar a inovação ou até mesmo ser um componente importante para entender como a economia é transformada devido à tecnologia. \\
	\midrule
	\cite{ziolo2022government} &
	\RaggedRight Cita que na União Europeia (até 2020), observou-se a correlação observada entre o nível de desenvolvimento do governo eletrônico e as áreas ambiental, social e econômica parece ser de grande importância, pois implica que a digitalização dos processos administrativos pode ter um impacto real no desenvolvimento sustentável, promovendo, assim, mudanças positivas em todas as suas três esferas. \\
	\midrule
	\cite{yamarik2023does} &
	\RaggedRight Cita que em sua pesquisa examinou a relação entre governo eletrônico e corrupção nos estados dos Estados Unidos encontraram que o governo eletrônico aumentou tanto as condenações por corrupção, quanto a percepção de corrupção. \\
	\midrule
	\cite{sugiarti2024effect} &
	\RaggedRight Esclarece que, baseado nos resultados estatísticos da testagem, o estudo providenciou evidências empíricas que o governo eletrônico teve uma influência negativa na corrupção. \\
\end{longtable}

Como exposto pela tabela \ref{tab:autores_beneficios_egov}, percebe-se quão benéfico é o governo eletrônico tanto para os governos, quanto para o povo. Dentre os benefícios, destaca-se a participação social.

Contudo, para \cite{de2020governo} o foco das políticas de governo eletrônico, em geral, permanece o mesmo: aprimorar processos internos de  trabalho, sem alterações significativas na cultura e na lógica burocráticas sobre as quais se estruturam as relações que se estabelecem entre a administração pública e os cidadãos.

De forma contínua, para \cite{cristovam2020governo} a Administração Pública brasileira tem usado as TIC no incremento de suas rotinas burocráticas. Há, ainda, o crescente uso dessas tecnologias na promoção do acesso à informação aos cidadãos. Mas ambos são usos na esteira do dito Governo eletrônico.

Consequentemente, conforme \cite{cristovam2020governo}, para se distanciar do governo eletrônico e poder implementar o governo digital, deve ser na qualidade de disrupção em relação ao Governo eletrônico, pois não almeja somente o emprego incremental de TICs e viabilização do acesso à informação, mas vai além, corporificando direitos sociais por intermédio do espaço digital.

As TIC podem contribuir para a inovação e o fomento da presta-
ção de serviços públicos adequados e atuais para todos os cidadãos, com-
portando as dimensões democrática e social impostas pela ordem jurídica
constitucional vigente. Os chamados e-Serviços Públicos abarcam a pres-
tação de serviços via internet, sistemas de imagem, radiodifusão, teleco-
municações, integração eletrônica gerencial de ações públicas, centrais
de atendimento e call center, afora a chamada Interface de Programa de
Aplicativos (do inglês, Application Programming Interface – API).

Importa indicar que a disponibilidade de direitos sociais por meio dos e-Ser-
viços Públicos dependerá de estudos específicos, com fito de determinar
qual ferramenta tecnológica será mais adequada.
A lógica dos e-Serviços Públicos, no esteio do Governo digital,
deve obedecer aos princípios da (i) eficiência, almejando a máxima satis-
fação do cidadão usuário; (ii) universalidade, maximizando a abrangência
da disponibilidade de tais serviços; e (iii) atualidade, garantindo que os
avanços tecnológicos, na comunicação e informação, sejam instrumentais
para as ações públicas de corporificação dos direitos sociais.
Por derradeiro, não se pode desconsiderar os desafios para a con-
cretização dessa mudança de paradigmas, em especial no ambiente pú-
blico e, em especial, na conjuntura do Governo digital e dos e-Serviços
Públicos. Mas, também, inviável olvidar a centralidade do nosso compro-
misso com a promoção da inclusão digital, de forma a aplacar a exclusão,
sobretudo a população mais pobre e vulnerabilizada, razão primeira das
mais variadas políticas públicas sociais e que deveria ser a primeira preo-
cupação governamental.

Um panorama que evidencia o destacado potencial das TICs para
contribuir na implementação de indicadores, métricas, levantamento de
dados e estudos técnicos capazes de apontar as mais adequadas tecnolo-
gias para cada direito social/respectivo serviço público, possibilitando ao
gestor público a melhor tomada de decisão no sentido de instrumentalizar
a ação administrativa e experienciar na atividade de gestão, com vistas à
efetiva encampação do complexo mosaico de objetivos constitucionais fonte última de legitimação do Poder Público e que deveria ser o prumo
balizador de todas as ações governamentais.



\subsubsection{governo digital}

\cite{martins2018war} argumenta que as ferramentas de governo digital promovem transparência, responsabilização e acesso melhorado à informação. E que os resultados encontrados pelo autor indicam claramente que níveis mais altos de governo eletrônico estão associados a melhores resultados no combate à corrupção.

\cite{veiga2016digital} afirma que o uso de governo digital e serviços públicos online têm um grande potencial de reduzir o fardo administrativo, bem como, promover inovação e crescimento econômico. Além de contribuir com a diminuição das atividades da economia informal, aumentando a quantidade de pessoas que pagam impostos e reduzing a corrupção.

\cite{veiga2016digital} afirma que o governo digital não é apenas sobre tecnologia, é sobre uma operação multifacetada  que requer uma abordagem multidisciplinar e disciplina científica.

\cite{alenezi2022understanding} afirma que a transformação digital no governo ou no setor público refere-se ao engajamento diferente e inovativo e trabalho com as parte interessadas, desenvolvendo frameworks para os mecanismos de entrega de serviços eficientes e formação de novos relacionamentos.

\cite{alenezi2022understanding} afirma que sua pesquisa destaca que um ambiente efetivo e favorável, força de trabalho qualificada, liderança, políticas públicas e regulações são os fatores chave do sucesso que podem encorajar e facilitar a rápida adaptação da transformação digital nas organizações do setor público.

\cite{bounabat2017government} afirma que o governo digital baseia-se na divulgação aberta e sem precedentes de informações governamentais, aliada à troca em grande volume de informações altamente sensíveis e/ou pessoais entre agências governamentais e seus clientes. 

\cite{bounabat2017government} complementa que ao mesmo tempo, tendências digitais como computação em nuvem, mobilidade, mídias sociais, big data e inteligência artificial geram diversos desafios de segurança assustadores, bem como preocupações com a privacidade dos cidadãos. 

\cite{bounabat2017government} argumenta que o sucesso de um programa de governo digital depende de quão bem ele enfrenta esses desafios e de quão bem ele consegue lidar com inúmeras ameaças potenciais, que vão desde simples atos de hacking até o ciberterrorismo.