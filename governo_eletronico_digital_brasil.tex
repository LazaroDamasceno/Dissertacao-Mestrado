\chapter{Governo eletrônico e digital no Brasil}

\cite{rover2009introduccao} argumenta que a interação entre as novas tecnologias, a sociedade e o Poder Público emoldura um momento único do qual emergem, simultaneamente, desafios enormes e vantagens sociais incríveis. Neste contexto, o aparecimento do governo eletrônico é uma decorrência das velhas e novas demandas da sociedade.

Para \cite{rover2009introduccao}, governo eletrônico é uma infra-estrutura única de comunicação compartilhada por diferentes órgãos públicos a partir da qual a TIC é usada de forma intensiva para melhorar a gestão pública e o atendimento ao cidadão.

Adicionalmente, como é entendido por \cite{rover2009introduccao}, o objetivo do governo eletrônico é colocar o governo ao alcance de todos, ampliando a transparências das suas ações e incrementando a participação cidadã, almejando a universalização de serviços.

Diversos autores destacam o impacto positivo do governo eletrônico na sociedade. 

\cite{martins2022digital} indicam que suas estimativas de que um nível alto de governo eletrônico podem facilitar negócios pela diminuição do fardo das regulações em diversos áreas de negócio.

\subsubsection{governo eletrônico: corrupção}

\cite{tavares2022governo} afirma que a transparência pública, exteriorizada em rede especialmente pelo uso de portais digitais, precisa ir além da mera disponibilização da informação e se preocupar também com a forma que essa informação chega a público, de forma organizada, de linguagem simples e padronizada.

\cite{martins2018war} ressalta que os resultados encontrados em sua pesquisa indicam que legisladores dos países menos desenvolvidos não devem olhar para o governo eletrônico como a solução final para combater corrupção. No entanto, páises mais desenvolvidos aparentam ser os que conseguem explorar plenamente o governo eletrônico como uma ferramenta chave para combater corrupção.

\cite{kotenok2020government} conclui que o impacto do governo eletrônico pode impulsionar a inovação ou até mesmo ser um componente importante para entender como a economia é transformada devido à tecnologia.

\cite{ziolo2022government} cita que na União Europeia (até 2020), observou-se a correlação observada entre o nível de desenvolvimento do governo eletrônico e as áreas ambiental, social e econômica parece ser de grande importância. Essa correlação implica que a digitalização dos processos administrativos pode ter um impacto real no desenvolvimento sustentável, promovendo, assim, mudanças positivas em todas as suas três esferas.

\cite{yamarik2023does} cita que em sua pesquisa examinou a relação entre governo eletrônico e corrupção nos estados dos Estados Unidos encontraram que o governo eletrônico aumentou tanto as condenações por corrupção, quanto a percepção de corrupção.

\cite{sugiarti2024effect} esclarece que, baseado nos resultados estatísticos da testagem, o estudo providenciou evidências empíricas que o governo eletrônico teve uma influência negativa na corrupção.

\subsubsection{governo digital}

\cite{martins2018war} argumenta que as ferramentas de governo digital promovem transparência, responsabilização e acesso melhorado à informação. E que os resultados encontrados pelo autor indicam claramente que níveis mais altos de governo eletrônico estão associados a melhores resultados no combate à corrupção.

\cite{veiga2016digital} afirma que o uso de governo digital e serviços públicos online têm um grande potencial de reduzir o fardo administrativo, bem como, promover inovação e crescimento econômico. Além de contribuir com a diminuição das atividades da economia informal, aumentando a quantidade de pessoas que pagam impostos e reduzing a corrupção.

\cite{veiga2016digital} afirma que o governo digital não é apenas sobre tecnologia, é sobre uma operação multifacetada  que requer uma abordagem multidisciplinar e disciplina científica.

\cite{alenezi2022understanding} afirma que a transformação digital no governo ou no setor público refere-se ao engajamento diferente e inovativo e trabalho com as parte interessadas, desenvolvendo frameworks para os mecanismos de entrega de serviços eficientes e formação de novos relacionamentos.

\cite{alenezi2022understanding} afirma que sua pesquisa destaca que um ambiente efetivo e favorável, força de trabalho qualificada, liderança, políticas públicas e regulações são os fatores chave do sucesso que podem encorajar e facilitar a rápida adaptação da transformação digital nas organizações do setor público.

\subsubsection{políticas públicas}

\cite{tavares2022governo} ressalta que a \href{https://www.planalto.gov.br/ccivil_03/constituicao/constituicao.htm}{Constituição Federal de 1988} fixou a cidadania como fundamento da República, tendo a participação e o controle papéis essenciais ao bom funcionamento do Estado, da Democracia e da Administração Pública, a partir da concepção de cidadania e democracia participativa.

\cite{tavares2022governo} argumenta que o controle social possui estreita ligação com as políticas públicas, pois, a partir do seu exercício, em todas as etapas do ciclo, desde a formulação até a avaliação, confere-se maior legitimidade e eficiência aos resultados dos objetivos, metas e diretrizes fixadas pelos planos, programas e ações dentro do conjunto de políticas públicas.

\cite{tavares2022governo} afirma que as políticas públicas são a forma como se resolve os problemas da sociedade e o controle social é a forma como o cidadão interage, fiscaliza e questiona as soluções definidas para esses problemas. 
