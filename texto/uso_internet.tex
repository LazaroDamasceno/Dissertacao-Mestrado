\section{Uso de internet}

O uso da internet tem crescido exponencialmente desde o ano 2000. Conforme a figura \ref{fig:individuals_using_internet_itu}, iniciando em menos de 2\% em 2000, o uso de internet atingiu um índice de mais de 60\% em 2024 globalmente. 

\begin{figure}[ht]
    \centering
    \caption{Uso pessoal de internet pelo mundo (2000-2024)}
    \includegraphics[width=0.78\linewidth]{figuras/internet/individuals_using_internet_itu.png}
    \label{fig:individuals_using_internet_itu}
    \\ \footnotesize{Fonte: ITU (2024)}
\end{figure}

O crescimento do uso da internet no Brasil foi tão exponenencial quanto a tendência do crescimento global. O notório crescimento no Brasil está representado na figura \ref{fig:crescimento_internet_brasil_itu}.

\begin{figure}[ht]
    \centering
    \caption{Crescimento do uso de internet no Brasil (2000-2023)}
    \includegraphics[width=1\linewidth]{figuras/internet/lineplot_uso_internet_brasil_itu.png}
    \label{fig:crescimento_internet_brasil_itu}
    \footnotesize{Fonte: ITU (2024)}
\end{figure}

A linha de tendência do crescimento demonstra que em 2000 o uso de internet no Brasil era um pouco mais do que 0\% de 2020, a porcentagem ultrapassou os 80\%. A figura \ref{fig:uso_internet_brasil_itu} mostra as porcentagens por ano desde 2000 até 2023. Embora houve dois retroceseos em 2021 e 2022, consecutivamente, fazendo o índice atingir os 80\%, ele ultrapassou a faixa dos 81\%, tal como era em 2020.

\begin{figure}[ht]
    \centering
    \caption{Uso de internet no Brasil (2000-2023)}
    \includegraphics[width=1\linewidth]{figuras/internet/barplot_uso_internet_brasil_itu.png}
    \label{fig:uso_internet_brasil_itu}
    \footnotesize{Fonte: ITU (2024)}
\end{figure}

Visando entender melhor o crescimento do uso de internet no Brasil, fez-se uma análise em que se comparava o índice do uso de internet no Brasil, a população total do Brasil e extensão territorial do país nos de 2000 até 2023 tem um coefieciente de correlação de Pearson positiva e forte. A figura \ref{fig:internet_correlacao} demonstra o resultado da análise.

\begin{figure}[ht]
    \centering
    \caption{Coeficiente de correlação de Pearson: índice do uso de internet, população total e extensão territorial}
    \includegraphics[width=1\linewidth]{figuras/internet/correlacao.png}
    \label{fig:internet_correlacao}
    \footnotesize{Fonte: Fonte: ONU (2025) e ITU (2024)}
\end{figure}

\newpage
O coefieciente de correlação de Pearson mais positivo e forte foi entre \texttt{Uso de internet} e \texttt{População total}. De maneira contraria, \texttt{Extensão territorial} tem um coeficiente negativo com \texttt{Uso de internet} e \texttt{População total}. 

Portanto, será considerada apenas a relação \texttt{Uso de internet} e \texttt{População total}, pois o crescimento da população (até 2023) está altamente correlacionado com o crescimento do uso de internet no Brasil.

Conforme exposto pela figura \ref{fig:internet_correlacao}, o crescimento populacional (até 2023) tem forte correlação com o uso de  internet no Brasil. Assim, percebe-se tal correlação ao analisar o gráfico de crescimento projetado pela ONU, presente na figura \ref{fig:populacao_brasil}.

\begin{figure}[ht]
    \centering
    \caption{Coeficiente de correlação de Pearson: índice do uso de internet, população total e extensão territorial}
    \includegraphics[width=1\linewidth]{figuras/populacao/populacao_brasil.png}
    \label{fig:populacao_brasil}
    \footnotesize{Fonte: Fonte: ONU (2025)}
\end{figure}

\newpage
Acrescenta-se que é importante analisar o contexto de crescimento populacional em que o coefieciente de correlação entre \texttt{Uso de internet} e \texttt{População total} da \ref{fig:internet_correlacao} está inserido. 

Percebe-se que até quase 2050, a população brasileira crescerá, e depois desse ano, começara a reduzir. Assim, até 2023, ano em o uso de internet no Brasil foi registrada pela ITU, a correlacao é alta entre uso de internet e a população total.

Enquando a possível mudança no resultado do coefieciente de correlação entre \texttt{Uso de internet} e \texttt{População total} da figura \ref{fig:internet_correlacao} não mudar, os dados sobre uso de internet serão focados no Brasil. Foram escolhidos os dados da pesquisa TIC Domicílios da \texttt{Cetic.br} da UNESCO (2005-2024) para representar a realidade brasileira. 

A representação da realidade nacional será dividida nas 5 regiões presentes na figura \ref{fig:regioes_brasil}. Haman (2017) argumenta que divindindo um país em regiões, torna-se possível compreender suas diferenças, de modo a conheçe-lo.

%HAMAM, Adma. Cinco faces do Brasil. Retratos, Rio de Janeiro, v. 6, p. 9-12, dez. 2017. Disponível em: https://agenciadenoticias.ibge.gov.br/media/com_mediaibge/arquivos/3ee63778c4cfdcbbe4684937273d15e2.pdf. Acesso em: 13 jul. 2025.

\begin{figure}[ht]
    \centering
    \caption{Regiões do Brasil}
    \includegraphics[width=1\linewidth]{figuras/regioes/regioes_brasil.PNG}
    \label{fig:regioes_brasil}
    \footnotesize{Fonte: Fonte: Haman (2017)}
\end{figure}
