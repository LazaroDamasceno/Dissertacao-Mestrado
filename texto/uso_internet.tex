\section{Uso de internet}
\label{uso_internet}

O uso da internet tem crescido exponencialmente desde o ano 2000. Conforme a figura \ref{fig:individuals_using_internet_itu}, iniciando em menos de 2\% em 2000, o uso da internet atingiu um índice de mais de 60\% em 2024 globalmente. 

\begin{figure}[H]
    \centering
    \caption{Uso pessoal de internet pelo mundo (2000-2024)}
    \includegraphics[width=0.78\linewidth]{figuras/internet/individuos_uso_internet_itu.png}
    \label{fig:individuals_using_internet_itu}
    \\ \footnotesize{Fonte: \cite{ITU_uso_internet_mundo}.}
\end{figure}

O crescimento do uso da internet no Brasil foi tão exponencial quanto a tendência do crescimento global. O notório crescimento no Brasil está representado na figura \ref{fig:crescimento_internet_brasil_itu}.

\begin{figure}[H]
    \centering
    \caption{Crescimento do uso de internet no Brasil (2000-2023)}
    \includegraphics[width=1\linewidth]{figuras/internet/lineplot_uso_internet_brasil_itu.png}
    \label{fig:crescimento_internet_brasil_itu}
    \footnotesize{Fonte: \cite{ITU_crescimento_uso_internet_brasil}.}
\end{figure}

A linha de tendência do crescimento demonstra que em 2000 o uso de internet no Brasil era um pouco mais do que 0\%; em 2020, a porcentagem ultrapassou os 80\%. A figura \ref{fig:uso_internet_brasil_itu} mostra as porcentagens por ano desde 2000 até 2023. Apesar de pequenas variações em 2021 e 2022, fazendo o índice atingir os 80\%, ultrapassou a faixa dos 81\%.

\begin{figure}[H]
    \centering
    \caption{Uso de internet no Brasil (2000-2023)}
    \includegraphics[width=1\linewidth]{figuras/internet/barplot_uso_internet_brasil_itu.png}
    \label{fig:uso_internet_brasil_itu}
    \footnotesize{Fonte: \cite{ITU_uso_internet_brasil}.}
\end{figure}

Visando entender melhor o crescimento do uso de internet no Brasil, fez-se uma análise em que se comparava o uso de internet no Brasil, a população total do Brasil e a extensão territorial do país. De 2000 até 2023, tem um coeficiente de correlação de Pearson positivo e forte. A figura \ref{fig:internet_correlacao} demonstra o resultado da análise.

\begin{figure}[H]
    \centering
    \caption{Coeficiente de correlação de Pearson: uso de internet, população brasileira e extensão territorial}
    \includegraphics[width=1\linewidth]{figuras/internet/correlacao.png}
    \label{fig:internet_correlacao}
    \footnotesize{Fonte: baseado em \cite{ONU_populacao_paises}, \cite{pnda_continua_anual_2016_2023} e \cite{ONU_tamanho_paises}.}
\end{figure}

O coeficiente de correlação de Pearson varia entre -1 e 1. O sinal indica a direção da correlação (negativa ou positiva) enquanto que o valor indica a magnitude. Quanto mais perto de 1 mais forte é o nível de associação linear entre as variáveis. Quanto mais perto de zero, menor é o nível de associação. Em particular, uma correlação de valor zero significa que as variáveis são ortogonais entre si (ausência de
correlação) \cite{sousa2019coeficiente}.

O coeficiente de correlação de Pearson mais positivo e forte foi entre \texttt{uso de internet} e \texttt{População brasileira}. De maneira contrária, \texttt{Extensão territorial} tem um coeficiente negativo com \texttt{uso de internet} e \texttt{População brasileira}. 

Portanto, será considerada apenas a relação \texttt{uso de internet} e \texttt{População brasileira}, pois o crescimento da população (até 2023) está altamente correlacionado com o crescimento do uso de internet no Brasil.

Conforme exposto pela figura \ref{fig:internet_correlacao}, o crescimento populacional (até 2023) tem forte correlação com o uso de internet no Brasil. Assim, percebe-se tal correlação ao analisar o gráfico de crescimento projetado pela ONU, presente na figura \ref{fig:populacao_brasil}.

\begin{figure}[H]
    \centering
    \caption{População projetada do Brasil pelo IBGE (2000-2070)}
    \includegraphics[width=1\linewidth]{figuras/populacao/populacao_brasil.png}
    \label{fig:populacao_brasil}
    \footnotesize{Fonte: \cite{IBGE_populacao_brasil}.}
\end{figure}

Acrescenta-se que é importante analisar o contexto de crescimento populacional em que o coeficiente de correlação entre \texttt{uso de internet} e \texttt{População total} da \ref{fig:internet_correlacao} está inserido. 

Percebe-se que até 2048, a população brasileira crescerá, e depois desse ano, começará a reduzir. Assim, até 2023, último ano em que o uso de internet no Brasil foi registrado pelo ITU, a correlação é alta entre uso de internet e a população total.

Como o período de mudança demográfica ainda não se concretizou, haverá o funilamento do contexto internacional para o brasileiro. Como consequência, foram escolhidos os dados de pesquisa da PNAD Contínua Anual 2016-2019, 2021-2023. Os dados da PNAD serão extraídos da tabela 6794 (\href{Htps://sidra.ibge.gov.br/tabela/6794}{Pessoas de 10 anos ou mais de idade, por sexo e grupo de idade}). Nas análises, sexo e as faixa etárias de 10-13 anos e 14-17 anos não serão levadas em consideração.

A representação da realidade nacional será dividida nas 5 regiões presentes na figura \ref{fig:regioes_brasil}. \cite{HAMAM_2017} argumenta que, dividindo um país em regiões, torna-se possível compreender suas diferenças, de modo a conhecê-lo.

\begin{figure}[H]
    \centering
    \caption{Regiões do Brasil}
    \includegraphics[width=0.75\linewidth]{figuras/regioes/regioes_brasil.PNG}
    \label{fig:regioes_brasil}
    \\ \footnotesize{Fonte: \cite{HAMAM_2017}.}
\end{figure}

Inspirando-se no argumento de \cite{HAMAM_2017}, criou-se o gráfico de distribuição presente na figura \ref{fig:distribuicao_uso_internet_regioes}.

\begin{figure}[H]
    \centering
    \caption{Distribuição do uso de internet pelas regioes}
    \includegraphics[width=1\linewidth]{figuras/internet/distribuicao_uso_internet_regioes.png}
    \label{fig:distribuicao_uso_internet_regioes}
    \footnotesize{Fonte: \cite{pnda_continua_anual_2016_2023}.}
\end{figure}

Note-se que todas as regiões seguem uma tendência de uso de internet, de modo que mesmo que o número de usuários aumente, uma região não ultrapassará outra. Há 4 classificações, pois as Regiões Centro-Oeste e Região Norte têm uma o índice mais baixo no uso nos anos de 2016-2019 e 2021-2023, estando na 4a classificação. A 3a classificação é ocupada pela Região Sul, seguido da Região Nordeste. A Região Sudeste ocupa a 1a classificação.

Para entender a figura \ref{fig:distribuicao_uso_internet_regioes}, foi feita um análise do coeficiente de correlação de Pearson com o índice de uso de internet com a população e densidade populacional de cada região no ano de 2022 com base nos dados do Censo e da PNAD Contínua Anual 2022.

A figura \ref{fig:densidade_demografica} contém o mapa do Brasil com a densidade demográfica de cada estado.

\begin{figure}[H]
    \centering
    \caption{Censo 2022: densidade demográfica}
    \includegraphics[width=1\linewidth]{figuras/populacao/censo_2022_densidade_demografica.png}
    \label{fig:densidade_demografica}
    \footnotesize{Fonte: baseado em \cite{IBGE_censo_2022}.}
\end{figure}

A escolha da densidade demográfica foi feita em razão de um questionamento ao analisar a figura \ref{fig:densidade_demografica}, pois se a correlação entre índice de uso de internet e a população é alta, qual seria o resultado se a densidade demográfica fosse usada como critério de análise do coeficiente de correlação de Pearson.

\begin{figure}[H]
    \centering
    \caption{Coeficiente de correlação de Pearson: índice de uso de internet e densidade populacional das regiões em 2022}
    \includegraphics[width=1\linewidth]{figuras/internet/correlacao2.png}
    \label{fig:correlacao2}
    \footnotesize{Fonte: baseado em \cite{IBGE_censo_2022} e \cite{pnda_continua_anual_2016_2023}.}
\end{figure}

Como já foi provado que há correlação entre o uso de internet e o crescimento da população brasileira (figura \ref{fig:internet_correlacao}), testou-se a correlação entre o uso de internet e a densidade demográfica das regiões.

O resultado foi uma correlação positiva e alta entre o uso de internet e a densidade demográfica. Devido a isso, e complementando com o teor da figura \ref{fig:densidade_demografica}, confirmou-se que o uso de internet é mais alto nos estados com alta densidade demográfica, ou seja, os que têm litoral, com exceção do Piauí, sendo Minas Gerais e Goiás os únicos estados não costeiros a ter alta densidade demográfica.
