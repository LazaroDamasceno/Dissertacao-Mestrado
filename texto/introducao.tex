\chapter{Introdução}

Serão analisados os sistemas Sistema Eletrônico de Informações, Creta, Processo Judicial Eletrônico e a Plataforma Digital do Poder Judiciário Brasileiro como medidas de digitalização e modernizadores do Poder Judiciário. 

Cada sistema representa tentativas de mudanças que foram buscadas para otimizar o funcionamento administrativo e judicial dos órgãos do poder judicante. Adicionalmente,
a abordagem é quantitativa, com foco exploratório. 

Para a análise do impacto da digitalização, foi escolhido o \textbf{E-Government Development Index} do Organização das Nações Unidas como parâmetro de medição da efetividade da digitalização dos serviços administrativos e judiciais do Poder Judiciário.

A análise dos dados para a comprovação da efetividade do impacto do uso dos sistemas Sistema Eletrônico de Informações, Creta, Processo Judicial Eletrônico e a Plataforma Digital do Poder Judiciário Brasileiro foi proporcionada pela linguagem de programação \textbf{Python} via suas bibliotecas \textbf{Pandas}, \textbf{GeoPandas}, \textbf{Matplotlib} e \textbf{Seaborn}.

Da análise, foram extraídos mapas coropléticos, gráficos de barra, de caixa e de linha, de modo que cada tipo de gráfico mostrará aspectos diferentes de como o EGDI tem impacto no Brasil, e por extensão, no Poder Judiciário.