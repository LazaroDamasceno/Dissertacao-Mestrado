\chapter{Digitalização do Poder Público Brasileiro}

\cite{tavares2022governo} afirma que as políticas públicas são a forma como se resolve os problemas da sociedade e o controle social é a forma como o cidadão interage, fiscaliza e questiona as soluções definidas para esses problemas. 

\cite{rover2009introduccao} argumenta que a interação entre as novas tecnologias, a sociedade e o Poder Público emoldura um momento único do qual emergem, simultaneamente, desafios enormes e vantagens sociais incríveis. Neste contexto, o aparecimento do governo eletrônico é uma decorrência das velhas e novas demandas da sociedade.

Para \cite{rover2009introduccao}, governo eletrônico é uma infra-estrutura única de comunicação compartilhada por diferentes órgãos públicos a partir da qual a TIC é usada de forma intensiva para melhorar a gestão pública e o atendimento ao cidadão.

Adicionalmente, como é entendido por \cite{rover2009introduccao}, o objetivo do governo eletrônico é colocar o governo ao alcance de todos, ampliando a transparências das suas ações e incrementando a participação cidadã, almejando a universalização de serviços.

\cite{singh2007country} projeta que a maturidade do governo eletrôncio pode ser considerada razoavelmente dependente de como está o estado da infraestrutura de TIC, em razão da sua capacidade de limitar o acesso aos serviços públicos digitais. 

Para \cite{singh2007country}, países com PIB per capita altos estão em melhor posição de dispor de infraestruturas difundidas, alta qualidade e físicas de TIC. Com altos níveis de acesso às TIC, os cidadãos tem uma tendência maior de usar serviços públicos digitais.

Quando os cidadãos passam a adotar os serviços públicos digitais, segundo \cite{singh2007country}, facilita ao poder público a transação completa dos serviços públicos presenciais para os digitais. A referida mudança pode ajudar na economia de recursos públicos, definindo um círculo virtuoso que justifica os investimentos em governo eletrônico.

\section{Benefícios do governo eletrônico}

Diversos autores destacam o impacto positivo do governo eletrônico na sociedade.  \cite{martins2018war} cita que os resultados encontrados indicam claramente que níveis mais altos de governo eletrônico estão associados a melhores resultados no combate à corrupção. Para \cite{kotenok2020government}, o impacto do governo eletrônico pode impulsionar a inovação ou até mesmo ser um componente importante para entender como a economia é transformada   devido à tecnologia.

\cite{martins2022digital} um nível alto de governo eletrônico podem facilitar negócios pela   diminuição do fardo das regulações em diversas áreas de negócio. \cite{sugiarti2024effect} sua pesquisa examinou a relação entre governo eletrônico e corrupção nos   estados dos Estados Unidos encontraram que o governo eletrônico aumentou   tanto as condenações por corrupção, quanto a percepção de corrupção.

Para \cite{ziolo2022government}, na União Europeia (até 2020) observou-se a correlação observada entre o nível de desenvolvimento do governo eletrônico e as áreas ambiental, social e econômica parece ser de grande importância, pois implica que a digitalização dos processos administrativos pode ter um impacto real no desenvolvimento   sustentável, promovendo, assim, mudanças positivas em todas as suas três esferas.

\section{Foco do governo eletrônico}

Contudo, para \cite{de2020governo} o foco das políticas de governo eletrônico, em geral, permanece o mesmo: aprimorar processos internos de  trabalho, sem alterações significativas na cultura e na lógica burocráticas sobre as quais se estruturam as relações que se estabelecem entre a administração pública e os cidadãos.

Assim, para \cite{cristovam2020governo} a Administração Pública brasileira tem usado as TIC no incremento de suas rotinas burocráticas. Há, ainda, o crescente uso dessas tecnologias na promoção do acesso à informação aos cidadãos. Mas ambos são usos na esteira do dito Governo eletrônico.

Consequentemente, conforme \cite{cristovam2020governo}, para se distanciar do governo eletrônico e poder implementar o governo digital, pois não se deve almeja somente o emprego incremental de TICs e a viabilização do acesso à informação, mas vai além, corporificando direitos sociais por intermédio do espaço digital.

Nesse sentido, quando \cite{cristovam2020governo} afirma que as TIC podem contribuir para a inovação e o fomento da prestação de serviços públicos adequados e atuais para todos os cidadãos, comportando as dimensões democrática e social impostas pela ordem jurídica constitucional vigente, há convergência com a ideia expressa por \cite{kotenok2020government} com os autores anteriores.

No dado contexto, \cite{alenezi2022understanding} afirma que sua pesquisa destaca que um ambiente efetivo e favorável, força de trabalho qualificada, liderança, políticas públicas e regulações são os fatores-chave do sucesso que podem encorajar e facilitar a rápida adaptação da transformação digital nas organizações do setor público.

\section{Entendendo o governo eletrônico no Brasil}
	
Como forma de entender o uso do governo eletrônico no Brasil, optou-se por \cite{tic_domicilios_2024}, devido ao seu objetivo de mapear o acesso às TIC nos domicílios urbanos e rurais do país e as suas formas de uso por indivíduos de 10 anos de idade ou mais. E ao fato de que o uso de governo eletrônico ser uma das suas áreas de investigação.

Em razão da continuidade das pesquisa TIC Domicílios desde 2005, escolheu-se o último de pesquisa (\textbf{2024}) da Cetic.BR. O tópico G foi o escolhido. Dele serão usados todos os seus indicadores G1, G2, G2A, G3. 

\subsection{G1: indivíduos que utilizaram governo eletrônico nos últimos 12 meses}

O primeiro, o G1, revelou o percentual de uso de governo eletrônico por indivíduos, cujo resultado está presente na figura \ref{fig:mapa_coropletico_tic_domicilio_g1}.

\begin{figure}[H]
	\centering
	\caption{Indicador G1: Uso de governo eletrônico por região do Brasil (em \%)}
	\includegraphics[width=1\linewidth]{figuras/mapa_coropletico_tic_domicilios_2024_g1}
	\label{fig:mapa_coropletico_tic_domicilio_g1}
	\footnotesize{Fonte: \cite{tic_domicilios_2024_g1}.}
\end{figure}

A figura \ref{fig:mapa_coropletico_tic_domicilio_g1} representa os resultados do indicador G1. As regiões Sul e Sudeste são as regiões que mais usam o governo eletrônico, seguidas das regiões Centro-Oeste e Norte. Por último, está o Nordeste.

\subsection{G2: usuários de internet, por tipo De informações referentes a serviços públicos procuradas ou Serviços públicos realizados nos últimos 12 meses}

O indicador G2 complementa o G1 ao especificar quais grupos de funções de governo eletrônico foram os mais usados. O indicador G2 tem os seguintes critérios:

\begin{itemize}
	\item Documentos pessoais, como RG, CPF, passaporte ou carteira de trabalho (G2-1)
	Saúde pública, como agendamento de consultas, remédios ou outros serviços do sistema público de saúde (G2-2).
	\item Educação pública, como Enem, Prouni, matrículas em escolas ou universidades públicas (G2-3).
	\item Direito do trabalhador ou previdência social, como INSS, FGTS, seguro-desemprego, auxílio-doença ou aposentadoria (G2-4).
	\item Impostos e taxas governamentais, como declaração de imposto de renda, IPVA ou IPTU (G2-5).
	\item Polícia e segurança, como boletim de ocorrência, antecedentes criminais ou denúncias (G2-6).
	\item Transporte público ou outros serviços urbanos, como limpeza e conservação de vias, iluminação (G2-7).
\end{itemize}

As figuras seguintes detalham como cada critério é usado por região.

\begin{figure}[H]
	\centering
	\caption{Indicador G2: critérios 1 e 2 (em \%)}
	\includegraphics[width=1\linewidth]{figuras/mapa_coropletico_tic_domicilios_2024_g2_1_2.png}
	\label{fig:mapa_coropletico_tic_domicilios_2024_g2_1_2}
	\footnotesize{Fonte: \cite{tic_domicilios_2024_g2}.}
\end{figure}

Quando se trata do indicador G2-1, as regiões que mais buscaram serviços públicos relativos a documentos pessoais, como RG, CPF, passaporte ou carteira de trabalho foram as Norte, Sudeste e Sul;

Quando se trata do indicador G2-2, apenas o Nordeste foi a região que menos uso serviços públicos relativos à saúde pública, como agendamento de consultas, remédios ou outros serviços do sistema público de saúde.

\begin{figure}[H]
	\centering
	\caption{Indicador G2: critérios 3 e 4 (em \%)}
	\includegraphics[width=1\linewidth]{figuras/mapa_coropletico_tic_domicilios_2024_g2_3_4.png}
	\label{fig:mapa_coropletico_tic_domicilios_2024_g2_3_4}
	\footnotesize{Fonte: \cite{tic_domicilios_2024_g2}.}
\end{figure}

Quando se trata do indicador G2-3, as regiões que mais usam serviços públicos relativos à educação pública, como Enem, Prouni, matrículas em escolas ou universidades públicas foram a Norte e a Sudeste.

Quando se trata do indicador G2-4, as regiões que mais usar serviços públicos relativos ao direito do trabalhador ou previdência social, como INSS, FGTS, seguro-desemprego, auxílio-doença ou aposentadoria foram as Sudeste e a Sul.

\begin{figure}[H]
	\centering
	\caption{Indicador G2: critérios 5 e 6 (em \%)}
	\includegraphics[width=1\linewidth]{figuras/mapa_coropletico_tic_domicilios_2024_g2_5_6.png}
	\label{fig:mapa_coropletico_tic_domicilios_2024_g2_5_6}
	\footnotesize{Fonte: \cite{tic_domicilios_2024_g2}.}
\end{figure}

Quando se trata do indicador G2-5, apenas as regiões Sudeste e Sul foram as que mais usaram serviços públicos relativos a impostos e taxas governamentais, como declaração de imposto de renda, IPVA ou IPTU.

Quando se trata do indicador G2-6, apenas a região Sul foi a que mais usou serviços públicos relativos à polícia e segurança, como boletim de ocorrência, antecedentes criminais ou denúncias.

\begin{figure}[H]
	\centering
	\caption{Indicador G2: critério 7 (em \%)}
	\includegraphics[width=1\linewidth]{figuras/mapa_coropletico_tic_domicilios_2024_g2_7.png}
	\label{fig:mapa_coropletico_tic_domicilios_2024_g2_7}
	\footnotesize{Fonte: \cite{tic_domicilios_2024_g2}.}
\end{figure}

Quando se trata do indicador G2-7, as regiões Sudeste e Sul foram as únicas que mais usaram serviços públicos relativos a transporte público ou outros serviços urbanos, como limpeza e conservação de vias e iluminação.

\subsection{G2A: usuários de internet, por necessidade de deslocamento para finalizar o serviço público}

Complementar ao indicador G2, o indicador G2A detalha se o serviço público foi realizado, completamente ou parcialmente, na internet, e se apenas informações do serviço público foram procuradas na internet, incluídas as opções em que o questionado não respondeu ou não sabe, todos como subcritérios. 

O indicador G2A tem 7 critérios, conforme exposto abaixo:

\begin{itemize}
	\item Documentos pessoais, como RG, CPF, passaporte ou carteira de trabalho.
	Saúde pública, como agendamento de consultas, remédios ou outros serviços do sistema público de saúde.
	\item Educação pública, como Enem, Prouni, matrículas em escolas ou universidades públicas.
	\item Direito do trabalhador ou previdência social, como INSS, FGTS, seguro-desemprego, auxílio-doença ou aposentadoria.
	\item Impostos e taxas governamentais, como declaração de imposto de renda, IPVA ou IPTU.
	\item Polícia e segurança, como boletim de ocorrência, antecedentes criminais ou denúncias.
	\item Transporte público ou outros serviços urbanos, como limpeza e conservação de vias, iluminação.
\end{itemize}

Os subcritérios do indicador G2A são, segundo \cite{tic_domicilios_2024_g2a}:

\begin{itemize}
    \item Realizou serviço na Internet sem precisar ir até um posto (SC1);  \item Realizou parte do serviço na Internet, mas precisou ir a um posto para finalizar (SC2);
    \item Apenas procurou informações na Internet (SC3);
    \item Não sabe; e
    \item Não respondeu.
\end{itemize}

As figuras \ref{fig:mapa_coropletico_tic_domicilios_2024_g2a_1}, \ref{fig:mapa_coropletico_tic_domicilios_2024_g2a_2}, \ref{fig:mapa_coropletico_tic_domicilios_2024_g2a_3},
\ref{fig:mapa_coropletico_tic_domicilios_2024_g2a_4},
\ref{fig:mapa_coropletico_tic_domicilios_2024_g2a_5} contêm mapas coropléticos que demonstram os subcritérios dos indicadores do G2A, que não incluirão as opções \textbf{não respondeu} e \textbf{não sabe}.

\begin{figure}[H]
	\centering
	\caption{Indicador G2A: critério 1 (em \%)}
	\includegraphics[width=1\linewidth]{figuras/mapa_coropletico_tic_domicilios_2024_g2a_1.png}
	\label{fig:mapa_coropletico_tic_domicilios_2024_g2a_1}
	\footnotesize{Fonte: \cite{tic_domicilios_2024_g2a}.}
\end{figure}

No tocante ao SC1, as regiões Sudeste e Sul foram as regiões em que mais ocorreram serviços na internet sem precisar ir até um posto. 

No tocante ao SC2, a região Sudeste foi a única região em que mais foram realizados partes dos serviços na Internet, mas foi preciso ir a um posto para finalizar, seguida do Centro-Oeste e das regiões Norte e Sul.

No tocante ao SC3, as regiões Norte, Nordeste, Sudeste foram as regiões em que mais se procurou informações na internet, seguidas do Sul e do Centro-Oeste.

\begin{figure}[H]
	\centering
	\caption{Indicador G2A: critério 2 (em \%)}
	\includegraphics[width=1\linewidth]{figuras/mapa_coropletico_tic_domicilios_2024_g2a_2.png}
	\label{fig:mapa_coropletico_tic_domicilios_2024_g2a_2}
	\footnotesize{Fonte: \cite{tic_domicilios_2024_g2a}.}
\end{figure}

No tocante ao SC1, as regiões Sudeste e Sul foram as regiões em que mais ocorreram serviços na internet sem precisar ir até um posto.

No tocante ao SC2, a região Sudeste foi a região em que mais foram realizados partes dos serviços na Internet, mas foi preciso ir a um posto para finalizar, seguidas  das regiões Sul e Norte, e por fim, do Centro-Oeste.

No tocante ao SC3, a região Sudeste foi a região em que mais se procurou informações na internet, seguida do Nordeste e Norte, bem como, conjuntamente, o Centro-Oeste e o Sul.

\begin{figure}[H]
	\centering
	\caption{Indicador G2A: critério 3 (em \%)}
	\includegraphics[width=1\linewidth]{figuras/mapa_coropletico_tic_domicilios_2024_g2a_3.png}
	\label{fig:mapa_coropletico_tic_domicilios_2024_g2a_3}
	\footnotesize{Fonte: \cite{tic_domicilios_2024_g2a}.}
\end{figure}

No tocante ao SC1, a região Sul foi a região em que mais ocorreram serviços na internet sem precisar ir até um posto, sendo o Nordeste a região em que mais se foi presencialmente aos postos.

No tocante ao SC2, as regiões Norte, Sudeste e Sul foram as regiões em que mais foram realizados partes dos serviços na Internet, mas foi preciso ir a um posto para finalizar, seguidas do Nordeste e Centro-Oeste.

No tocante ao SC3, as regiões Norte e Sudeste em que mais se procurou informações na internet, seguidas do Nordeste e das regiões Centro-Oeste e Sul.

\begin{figure}[H]
	\centering
	\caption{Indicador G2A: critério 4 (em \%)}
	\includegraphics[width=1\linewidth]{figuras/mapa_coropletico_tic_domicilios_2024_g2a_4.png}
	\label{fig:mapa_coropletico_tic_domicilios_2024_g2a_4}
	\footnotesize{Fonte: \cite{tic_domicilios_2024_g2a}.}
\end{figure}

No tocante ao SC1, as regiões Sudeste e Sul foram as regiões em que mais ocorreram serviços na internet sem precisar ir até um posto, seguidas do Centro-Oeste e das regiões Norte e Nordeste.

No tocante ao SC2, a região Sul foi a região em que mais foram realizados partes dos serviços na Internet, mas foi preciso ir a um posto para finalizar, seguidas do Sudeste e Norte e das regiões Centro-Oeste e Nordeste.

No tocante ao SC3, a região Nordeste foi a região em que mais se procurou informações na internet, seguidas do Norte e Sudeste e da região Centro-Oeste.

\begin{figure}[H]
	\centering
	\caption{Indicador G2A: critério 5 (em \%)}
	\includegraphics[width=1\linewidth]{figuras/mapa_coropletico_tic_domicilios_2024_g2a_5.png}
	\label{fig:mapa_coropletico_tic_domicilios_2024_g2a_5}
	\footnotesize{Fonte: \cite{tic_domicilios_2024_g2a}.}
\end{figure}

No tocante ao SC1, a região Sul foi a região em que mais ocorreram serviços na internet sem precisar ir até um posto, seguidas do Sudeste, Centro-Oeste e das regiões Norte e Nordeste.

No tocante ao SC2, a região Sul foi a região em que mais foram realizados partes dos serviços na Internet, mas foi preciso ir a um posto para finalizar, seguidas do Centro-Oeste e Norte e das regiões Sudeste e Nordeste.

No tocante ao SC3, as regiões Norte e Nordeste foram a região em que mais se procurou informações na internet, seguidas do Sudeste, Sul e da região Centro-Oeste.

\begin{figure}[H]
	\centering
	\caption{Indicador G2A: critério 6 (em \%)}
	\includegraphics[width=1\linewidth]{figuras/mapa_coropletico_tic_domicilios_2024_g2a_6.png}
	\label{fig:mapa_coropletico_tic_domicilios_2024_g2a_6}
	\footnotesize{Fonte: \cite{tic_domicilios_2024_g2a}.}
\end{figure}

No tocante ao SC1, apenas a região Sul foi a região em que mais ocorreram serviços na internet sem precisar ir até um posto.

No tocante ao SC2, a região Norte foi a região em que mais foram realizados partes dos serviços na Internet, mas foi preciso ir a um posto para finalizar, seguidas do Centro-Oeste e Sudeste e das regiões Sul e Nordeste.

No tocante ao SC3, as regiões Norte e Sudeste foram a região em que mais se procurou informações na internet, seguidas do Nordeste e das regiões Centro-Oeste e Sul.

\begin{figure}[H]
	\centering
	\caption{Indicador G2A: critério 7 (em \%)}
	\includegraphics[width=1\linewidth]{figuras/mapa_coropletico_tic_domicilios_2024_g2a_7.png}
	\label{fig:mapa_coropletico_tic_domicilios_2024_g2a_7}
	\footnotesize{Fonte: \cite{tic_domicilios_2024_g2a}.}
\end{figure}

No tocante ao SC1, a região Sul foi a região em que mais ocorreram serviços na internet sem precisar ir até um posto, seguidas das regiões Sudeste, conjuntamente, o Centro-Oeste e o Norte, e por fim, o Nordeste.

No tocante ao SC2, apenas a região Centro-Oeste foi a região em que mais foram realizados partes dos serviços na Internet.

No tocante ao SC3, as regiões Norte e Sudeste foram a região em que mais se procurou informações na internet, seguidas do Nordeste e das regiões Centro-Oeste e Sul.

\subsection{G3: usuários de internet, por atividades de interação com autoridades públicas}

Terminando a análise do TIC Domicílios 2024, analisar-se-á o indicador G3. O indicador representa os usuários de internet, por atividades de interação com autoridades públicas.

A lista abaixo contém a descrição dos seus 3 critérios.  

\begin{itemize}
	\item Procurou informações oferecidas por sites de governo (C1).
	\item Realizou algum serviço público, como emitir documentos pela Internet, preencher e enviar formulários online ou pagar taxas e impostos pela Internet (C2).
	\item Não utilizou a Internet para realizar atividades de interação com autoridades públicas (C3).
\end{itemize}

Haja vista a lista acima, a figura \ref{fig:mapa_coropletico_tic_domicilios_2024_g3}  representa o percentual de usuários de internet, por atividades de interação com autoridades públicas. 

\begin{figure}[H]
	\centering
	\caption{Indicador G3: critérios (em \%)}
	\includegraphics[width=1\linewidth]{figuras/mapa_coropletico_tic_domicilios_2024_g3.png}
	\label{fig:mapa_coropletico_tic_domicilios_2024_g3}
	\footnotesize{Fonte: elaboração própria baseade em \cite{tic_domicilios_2024_g3}.}
\end{figure}

No tocante ao critério G3-C1, as regiões Sudeste e Sul foram as regiões em que mais houve procura de informações oferecidas pelo governo, seguidas do Centro-Oeste e Norte, e por fim, pelo Nordeste.

No tocante ao critério G3-C2, a região Sul foi a região em que foram realizados alguns serviços públicos, como emitir documentos pela internet, preencher e enviar formulários online ou pagar taxas e impostos pela internet. A região que menos usou serviços públicos foi a Nordeste, superada pelas Centro-Oeste, Norte e Sudeste.

No tocante ao critério G3-C3, o Nordeste foi a única região em que a internet não foi utilizada para interagir com as autoridades, seguido do Centro-Oeste e finalmente, o Sudeste e o Sul.

Como foi demonstrado por todas as figuras, nota-se como é notório o uso de governo eletrônico no Brasil. Tal resultado confirma a ideia de \cite{singh2007country}, que argumenta que o uso constante de governo eletrônico justifica sua existência, manutenção e evolução.

\section{Transação do governo eletrônico para o digital no Brasil}

Como expressado nos parágrafos anteriores, com as condições favoráveis, a transformação digital pode se tornar paupável, executável e planejável. Segundo \cite{mitkiewicz2024transformaccao}, a transformação digital pode ser entendida como o processo de utilização das tecnologias da informação e comunicação para gerar soluções visando resolver de forma inovadora e em larga escala os problemas do mundo.

De forma complementar, \cite{alenezi2022understanding} afirma que a transformação digital no governo ou no setor público refere-se ao engajamento diferente e inovador e o trabalho com as partes interessadas, desenvolvendo frameworks para os mecanismos de entrega de serviços eficientes e formação de novos relacionamentos.

No contexto dos parágrafos anteriores, surgem os governos digitais em substituição aos governos eletrônicos. \cite{veiga2016digital} afirma que, diferentemente do governo eletrônico, o governo digital não é apenas sobre tecnologia, é sobre uma operação multifacetada  que requer uma abordagem multidisciplinar e disciplina científica. 

\cite{bounabat2017government} complementa a ideia anterior. O auto cita que o governo digital baseia-se na divulgação aberta e sem precedentes de informações governamentais, aliada à troca em grande volume de informações altamente sensíveis e também pessoais entre agências governamentais e seus clientes. 

O governo digital traz diversos benefícios, além dos benefícios do governo eletrônico. \cite{martins2018war} argumenta que as ferramentas de governo digital promovem transparência, responsabilização e acesso melhorado à informação.

Outra vantagem é mencionada por \cite{veiga2016digital}. O autor afirma que o uso de governo digital e serviços públicos online têm um grande potencial de reduzir o fardo administrativo, bem como, promover inovação e crescimento econômico. Além de contribuir com a diminuição das atividades da economia informal, aumentando a quantidade de pessoas que pagam impostos e reduzindo a corrupção.

Para \cite{kreuz20184textordfeminine}, vive-se e assiste-se à chegada da 4ª Revolução Industrial, que imprime uma modificação substancial na forma pela qual as pessoas e os diversos sistemas se relacionam. 

Complementa \cite{kreuz20184textordfeminine} que o mundo jurídico e o poder estatal necessitam não apenas se adaptar, mas incorporar as tecnologias ao seu m\textit{modus operandi} como meio de implementar a participação social dos cidadãos no processo decisório. E assim, incorporar os preceitos reais de um constitucionalismo latino-americano.

\cite{kreuz20184textordfeminine} argumenta que o estudo de caso realizado mediante o Governo digital brasileiro fornece algumas respostas. O Brasil vem se adaptando e implementando as TICs nos seus processos de relação com a sociedade. A abertura de dados e transparência cresce a cada ano. 

\cite{kenosi2024industrial} em sua revisão da litaratura, indetificou que sua revião sistemática evidência que o potencial transformativo das tecnologias da Revolução Industrial 4.0 em melhorar os serviços de governo eletronico, focando na democrtização da administração pública vi a transparência melhorada, participação cidadã e entrega de serviços públicos.

No referido contexto, como marco legal da transição de governo eletrônico para digital  no Brasil, a Lei do Governo Digital, em seu artigo 1º, segundo \cite{l14129},  dispõe sobre princípios, regras e instrumentos para o aumento da eficiência da administração pública, especialmente por meio da desburocratização, da inovação, da transformação digital e da participação do cidadão.

\cite{carvalho2022nova} cita que foi com a Lei nº 14.129, de 29 de março de 2021, que o maior passo foi dado no sentido de implantação do governo digital. A propósito, apesar de ter se autointitulado “Lei do Governo Digital”, a citada legislação possui um campo de incidência ainda mais amplo.

Para \cite{carvalho2022nova}, no caso, o art. 1º da Lei do Governo Digital determina que a lei “dispõe sobre princípios, regras e instrumentos para o aumento da eficiência da administração pública”, e tal objetivo seria perseguido “especialmente por meio da desburocratização, da inovação, da  transformação digital e da participação do cidadão”. 

Para \cite{carvalho2022nova}, ademais, dentre os 26 princípios do governo digital e da eficiência pública (art. 3º), registre-se a presença tanto da “desburocratização, a modernização, o fortalecimento e a simplificação da relação do poder público com a sociedade, mediante serviços digitais”, como também do “incentivo à participação social no controle e na fiscalização da administração pública”.

Para \cite{carvalho2022nova}, a abertura de novos espaços de participação social na atividade administrativa do Estado trata-se de um importante passo para uma maior democratização da democracia, seja em relação ao aspecto quantitativo ou qualitativo, pois geralmente é a face da administração pública que apresenta o Estado ao cidadão.

\cite{carvalho2022nova} cita 4 aspectos do governo digital: 

\begin{itemize}
	\item Incremento da participação social pelo acesso dos 
cidadãos à informação e ao conhecimento
	\item A participação social digital gerando maior engajamento e 
empoderamento
	\item O governo digital aproximando a sociedade civil e o Estado
	\item Incremento da participação social viabilizada pelo monitoramento
\end{itemize}

Dos aspectos apresentados por \cite{carvalho2022nova}, será discutivo o aspecto do governo digital aproximando a sociedade civil e o Estado. Para \cite{carvalho2022nova}, um aspecto que precisa ser levado em consideração trata de uma forma de aproximação: a do Estado e da sociedade civil. Neste sentido, perceba-se que as novas tecnologias digitais não apenas permitem que a sociedade civil esteja informada, capacitada, engajada e empoderada. Elas também asseguram que o Estado possa melhor detectar qual são as aspirações dos cidadãos.

Complementarmente, para \cite{carvalho2022nova}, nesse momento de consolidação das redes sociais, vê-se a formação de uma opinião pública digital, que permite a qualquer cidadão expressar livremente suas opiniões e inclusive sugerir propostas de ação a serem adotadas pelo administrador público. O cidadão de hoje não pode apenas ser visto como um cliente da administração pública, mas sim um parceiro para a formulação e a execução de políticas estatais.

Para \cite{do2022governo}, a \text{Lei do Governo Digital}, propondo um modelo de governo digital que inaugure uma nova forma de relacionamento entre a Administração Pública e os destinatários de sua atuação, incorpora ferramentas de modificação na dinâmica tradicional regedora dessas mesmas relações.

Ainda para \cite{do2022governo}, em razão do argumento anterior, promove uma conciliação entre a racionalidade jurídica, que se encontra na regularidade do procedimento e na estabilidade das estruturas formais de organização e atuação, e a racionalidade da gestão, que tem por fonte de legitimidade a eficácia das ações desenvolvidas. 

\cite{do2022governo} elogia a mudança do paradigma legal introduzida pela \text{Lei do Governo Digital} como uma a iniciativa é de ser prestigiada, pois no alinhamento entre racionalidade jurídica e racionalidade da gestão tem-se a tradução de um direito fundamental à boa administração.

Porém, limitações para a implementação das políticas públicas de governo digital. \cite{reck2021transformaccao} cita que ainda falta muito a fazer, principalmente no que diz respeito ao processo de inclusão dos que não tem acesso aos serviços indispensáveis à busca de uma cidadania mais efetiva. 

\cite{reck2021transformaccao} complementa que o cenário do parágrafo anterior é composto por cerca de 30 milhões de pessoas, que não estão inseridos no processo democrático, em razão de não terem sido alcançados pelos serviços digitais.  Nos tempos atuais, as políticas públicas podem ser disseminadas por plataformas digitais, notadamente, pelo fato do avanço de serviços como o Governo Digital onde algumas ofertas estão cada mais exclusivas nesse meio, a citar o seguro-desemprego, meu SUS digital e previdência.

Assim, para  \cite{reck2021transformaccao}, vê-se que o exercício da cidadania para muitos, encontra obstáculos, faltando-lhes serem alcançados pelas TICs, bem como por uma educação voltada ao manejo da rede, como também, por lhes faltarem condições de acesso, especialmente à população mais vulnerável, como: negros, camponeses, ribeirinhos, povos originários, populações tradicionais, os de baixa renda; e os que mesmo tendo acesso, não tenham o necessário discernimento para fazê-lo com efetividade.

A Lei do Governo Digital nasceu do Projeto de Lei nº 7.843, de 2017. \cite{pl_lgd} cita como motivações para a proposição da inovação legislativa:

\begin{itemize}
    \item As críticas da qualidade ao atendimento do setor público.
    \item A precariedade e a falta de acesso a serviços públicos como fatores determinantes para o grave quadro de exclusão e desigualdade social que sempre marcou a sociedade brasileira. 
    \item  A simplificação das relações entre pessoas,
    sejam elas físicas ou jurídicas, com o poder público, tema essencial para o acesso a direitos básicos e, principalmente, para o desenvolvimento econômico.
    \item O excesso de exigências burocráticas, a baixa informatização, o ainda frágil acesso à informação, a falta de abertura das bases de dados públicos, a ausência de mecanismos de participação e inovação, além da corrupção, são alguns dos problemas que explicam a precariedade e ineficiência dos serviços públicos prestados
    nas três esferas da federação.
\end{itemize}

Nesse sentido, a Lei do Governo Digital, visando melhorar a administração pública, seu artigo 5º, conforme \cite{l14129}, serão utilizadas soluções digitais para a gestão de suas políticas finalísticas e administrativas e para o trâmite de processos administrativos eletrônicos.

Além disso, Lei do Governo Digital, segundo \cite{l14129}, determina sua aplicação às Administrações Direta e Indireta da União Federal e dos demais entes federados, desde que adotem os comandos da lei por meio de atos normativos próprios, vedada a aplicação da lei às empresas públicas e sociedades de economia mista, suas subsidiárias e controladas que não prestem serviço público.

Haja vista \cite{reck2021transformaccao}, o governo digital não se restringe à automação de processos e à disponibilização de serviços públicos on-line, busca avançar para um modelo de administração pública capaz de integrar as TICs a seus processos internos e aos cidadãos, buscando cumprir os papéis essenciais do Estado de forma mais eficiente, bem como restar serviços públicos mais qualificados. 

De forma complementar ao argumento anterior, \cite{lima2023governo}, com as inovações legislativas trazidas pela Lei do Governo Digital, especialmente com o enfoque em um modelo do Governo digital por plataforma, notadamente na esfera federal, a mudança está em sintonia com as mudanças tecnológicas, principalmente impulsionadas pela pandemia da Covid-19, guarda estrita sintonia com a ordem jurídica constitucional vigente e evidencia essa ordem de preocupação normativa da parte do Poder Público.

A maneira como o poder público se relaciona com a sociedade civil, sob a ótica do governo digital, segundo  \cite{lima2023governo}, é via as plataformas de governo digital, pois constituem realizações que buscam a aproximação entre Administração e cidadãos e cidadãs na esfera digital e, adicionalmente, são os meios pelos quais a atuação pública alcança suas finalidades.

Para \cite{de2020governo}, ao longo dos anos, a administração pública no Brasil se estruturou e foi moldada a partir de um amálgama entre uma concepção jurídica formalista, práticas burocráticas e uma generalizada cultura da desconfiança. 

\cite{de2020governo} complementa a ideia anterior citando as diversas faces 
conhecidas do modelo democrático:  

\begin{itemize}
    \item Interpretações e decisões baseadas em conceitos abstratos, ignorando as suas consequências práticas, defesa de ritos e formas como um fim em si mesmo, exigências de regularização desnecessárias.
    \item Um ambiente institucional que incentiva e premia o conservadorismo e a apatia de servidores e gestores públicos.
\end{itemize}

\cite{de2020governo} argumenta que as iniciativas de governo digital - sucessoras do governo eletrônico - pretendem, justamente, transformar a realidade do modelo burocrático inefetivo, eficaz  e ineficiente, mediante a instituição de serviços públicos digitais, que sejam mais simples, céleres e eficientes.

A implementação das iniciativas de governo digital trata-se, segundo \cite{de2020governo}, da construção de um novo paradigma  de administração pública, fundado sobre os princípios da transparência, da inovação e da confiança  segundo os quais o uso das tecnologias digitais pode e deve viabilizar: 

\begin{itemize}
    \item A ampliação do acesso às informações públicas e a simplificação de 
    mecanismos de prestação de contas e de interação entre a administração 
    pública e a sociedade, incluindo a instituição de novos mecanismos de 
    avaliação dos serviços.
    \item A efetiva e constante inovação, mediante a adoção de modelos 
    administrativos e jurídicos flexíveis, a admissibilidade controlada do risco, a relativa tolerância ao erro, o questionamento de práticas vigentes e a criação de incentivos para a experimentação e para a implementação de soluções criativas por parte de gestores públicos.
    \item Com base na arquitetura disponibilizada pelas tecnologias digitais, a constituição de novos modos de produção da confiança, por meio dos quais seja possível a redução de exigências burocráticas, bem como a garantia de maior simplicidade, celeridade, previsibilidade e segurança nas relações entre cidadãos e órgãos e entidades públicos.
\end{itemize}

Relativo ao primeiro tópico, o Brasil sanou o problema citado com a aprovação da Lei nº 12.527, de 2011 - Lei de Acesso à Informação e com a implementação dos portais da transparência do órgãos e Poderes dos Entes Federados.

O segundo tópico foi sanado pelo Lei do Governo Digital pela criação dos laboratórios de inovação. Para \cite{l14129}, laboratório de inovação é um espaço aberto à participação e à colaboração da sociedade para o desenvolvimento de ideias, de ferramentas e de métodos inovadores para a gestão pública, a prestação de serviços públicos e a participação do cidadão no exercício do controle sobre a administração pública.

O último e terceiro tópico foi sanado com a determinação legal de que apenas o CPF, para pessoas físicas, e o CNPJ, para pessoas jurídicas, como a única forma de identificação aceita pela administração pública, haja vista a Lei do Governo Digital (art. 28, \textbf{caput}), conforme exposto por \cite{l14129}: "Art. 28.  Fica estabelecido o número de inscrição no Cadastro de Pessoas Físicas (CPF) ou no Cadastro Nacional da Pessoa Jurídica (CNPJ) como número suficiente para identificação do cidadão ou da pessoa jurídica, conforme o caso, nos bancos de dados de serviços públicos, garantida a gratuidade da inscrição e das alterações nesses cadastros."

Independentemente dos benefícios apresentados pelos argumentos anteriores, considerando \cite{de2020governo} a ideia de que há diversos obstáculos que podem dificultar ou desvirtuar o sentido e os resultados das políticas de governo digital, dentre elas: o risco de digitalização de fachada e se foi realizada sem as devidas salvaguardas técnicas e jurídicas.

No tocante ao risco de digitalização de fachada, segundo \cite{de2020governo}, pode ocorrer a manutenção da lógica burocrática tradicional sob uma roupagem eletrônica, equívoco muitas vezes encontrado na administração pública brasileira. O segundo problema, a incorporação de tecnologias digitais pode gerar externalidades negativas, produzindo novos riscos e incertezas ou, ainda, abusos e violação de direitos.

Outros fatores são citados como barreiras para a implementação das políticas de governo digital, são para \cite{do2022governo}: 

\begin{itemize}
    \item No campo da resistência cultural, o investimento, obrigatoria
    mente, deve ser no treinamento e na formação das lideranças públicas 
    a conduzirem o processo.
    \item Na relação com o controle, as iniciativas associadas 
    ao governo digital devem se pautar, principalmente, pelo sempre pres
    tigiado vetor da transparência
\end{itemize}

No tocante ao primeiro tópico, \cite{do2022governo} afirma que o investimento, obrigatoriamente, deve ser no treinamento e na formação das lideranças públicas 
a conduzirem o processo. Educação digital deve ser a palavra de ordem dentro da Administração, para os seus próprios agentes, e em favor dos destinatários do governo digital. 

Como resultado da educação para mitigar a resistência à digitalização do poder público, segundo \cite{do2022governo}, ampliada a educação digital, o valor inerente ao governo de mesmo cariz resta autoevidente, e com isso a tendência é de mitigação da resistência a partir da perspectiva de constrição fiscal.

No tocante ao segundo e último tópico, \cite{do2022governo} argumenta que na relação com o controle, principalmente, pelo sempre prestigiado vetor da transparência. O problema não está em navegar em mares nunca dantes navegados, mas sim, em não ter clareza quanto às ondas que se possam ter pela frente.

\section{Lei do Governo Digital do Brasil sob à ótica de medidas internacionais}

As últimas edições dos GTMI e DGI foram em 2022 e 2023, respectivamente (até 27/08/2025).Embora as novas edições dos indicadores poderão ser publicadas ainda em 2025 ou mesmo em 2026, os indicadores são usados como medidas de análise secundárias. Como a Lei do Governo Digital foi aprovada em 2021, ano muito próximo de 2022 e 2023, avaliar-se-á como dois anos inciais consecutivos da referida lei foram avaliados pelos índices. Além dos GTMI e DGI, usar-se-ão relatórios da OCDE.

\subsection{GovTech Maturity Index}

As figuras \ref{fig:gtmi_2020_survey_map} e \ref{fig:gtmi_2022_survey_map} contêm mapas temáticos do GTMI no mundo de 2020 e 2022, em que há divisões de países por grupos.

\begin{figure}[H]
	\centering
	\caption{Estado do GTMI pelos grupos no mundo em 2020}
	\includegraphics[width=1\linewidth]{figuras/gtmi_2020_survey_map}
	\label{fig:gtmi_2020_survey_map}
	\footnotesize{Fonte: \cite{gtmi_2020_findings}.}
\end{figure}

\begin{figure}[H]
	\centering
	\caption{Estado do GTMI pelos grupos no mundo em 2022}
	\includegraphics[width=1\linewidth]{figuras/gtmi_2022_survey_map}
	\label{fig:gtmi_2022_survey_map}
	\footnotesize{Fonte: \cite{gtmi_2022}.}
\end{figure}

Nota-se como antes de 2021, o GTMI já posicionava o Brasil no grupo mais importante: o "A". Tal resultado foi mantido em 2022 na 2ª edição do GTMI. Para entender a importância do grupo "A" para o Brasil, destaque-se que a figura \ref{fig:definicao_grupos_gtmi} que contém um a tabela com a descrição dos grupos do GTMI.

\begin{figure}[H]
	\centering
	\caption{Grupos do GTMI e seus significados}
	\includegraphics[width=1\linewidth]{figuras/definicao_grupos_gtmi}
	\label{fig:definicao_grupos_gtmi}
	\footnotesize{Fonte: \cite{gtmi_2022}.}
\end{figure}

Conforme exposto pela figura \ref{fig:definicao_grupos_gtmi}, o grupo "A" representa o conjunto de países lideres em \textbf{GovTech} que demonstram avanções ou soluções inovadoras em todas as quatro áreas do GTMI.

De forma complementar, a figura \ref{fig:avg_gtmi_scores} contém a pontuação média dos grupos do GTMI de 2020 e 2022.

\begin{figure}[H]
	\centering
	\caption{Pontuação média dos grupos do GTMIde de 2020 e 2022}
	\includegraphics[width=1\linewidth]{figuras/avg_gtmi_scores}
	\label{fig:avg_gtmi_scores}
	\footnotesize{Fonte: \cite{gtmi_2022}.}
\end{figure}

Da figura \ref{fig:avg_gtmi_scores} extrai-se que o valor médio da pontuação do GTMI subiu, como também, os valores médios por grupo. 

Além disso, a figura \ref{fig:avg_gtmi_scores_groups} contém a pontuação média dos componentes dos grupos do GTMI. 

\begin{figure}[H]
	\centering
	\caption{Pontuação média dos componentes dos grupos do GTMI}
	\includegraphics[width=1\linewidth]{figuras/avg_gtmi_scores_groups}
	\label{fig:avg_gtmi_scores_groups}
	\footnotesize{Fonte: \cite{gtmi_2022}.}
\end{figure}

Conforme exposto pela figura \ref{fig:avg_gtmi_scores_groups}, as pontuações alcançadas pelas 4 áreas do GTMI do grupo "A" são as mais altas entre todos os grupos, e que, similar a degraus de uma escada, as pontuações alcançadas pelos demais grupos diminuem. Assim, como consequência do Brasil estar no grupo "A", \cite{gtmi_2022_latinamerica} destacou as boas práticas efetivadas pelo Brasil. 

Inicialmente, \cite{gtmi_2022_latinamerica} destacou as iniciativas da Secretaria de Governo Digital (SGD) como umas boas práticas analisadas na América Latina e Caribe pelas pesquisa do GTMI para o melhoramento do componente CGSI. 

Para \cite{gtmi_2022_latinamerica}, a SGD liderou em 2022 a implementação de uma abordagem multi-departamental para a transformação digital. Dentro medidas do Governo Federal, o site do Governo Digital apresentou os planos de transformação digital para os anos de 2021 e 2022, concomitantemente, as diretrizes atualizadas em 2022 para a adoção de serviços de computação em nuvem para a Administração Pública Federal.

Em segundo lugar, \cite{gtmi_2022_latinamerica} argumentou em relação às plataformas de serviços públicos digitais que o Brasil foi capaz de fortalecer os seus serviços públicos digitais por dois anos\footnote{Deve-se lembrar o texto foi escrito em 2022}. 

\cite{gtmi_2022_latinamerica} complementa que a criação da portal único gov.br em 2019, e a consecutiva digitalização e integração de serviços públicos aumentaram e consolidaram as interações digitais com os cidadãos e negócios, de modo que tal mudança reflete como a combinação de apoio política de alto nível, investimentos de qualidade consistente e forte engajamento de diferentes setores e níveis de governo pode gerar um efeito de salto na oferta de serviços públicos online eficazes.

Finalmente, outro aspecto analisado por \cite{gtmi_2022_latinamerica} foi o engajamento digital dos cidadãos. O autor afirma que tem sido membro do \textbf{Open Government Partnership} desde 2011 e tem o compromisso da publicação de dados governamentais e relatório para melhorar a abertura e transparência de relatórios fiscais.

Como exemplos de medidas de governo aberto, \cite{gtmi_2022_latinamerica} deu dois exemplos de políticas públicas de governo aberto do Brasil: a plataforma Participa+Brasil\footnote{cf. \url{https://www.gov.br/participamaisbrasil/pagina-inicial}} e o Portal de Dados Abertos\footnote{cf. \url{https://dados.gov.br/home}}. A primeira
permite aos cidadãos contribuir para a elaboração das políticas públicas e o melhoramento dos serviços públicos ao compartilhar \textbf{feedbacks}, sugestões e críticas construtivas, 

\cite{gtmi_2022_latinamerica} finaliza argumentando que a plataforma Participa+Brasil, além do citado \textit{a priori}, é onde o Poder Executivo publica as estatísticas do engajamento dos cidadãos e implementa as iniciativas para melhorar a representação dos grupos vulneráveis\footnote{É importante ressaltas que o texto de \cite{gtmi_2022_latinamerica} foi publicado em 2022. Portanto, mudanças podem ter ocorrido após 2022. No entanto, optou-e por \cite{gtmi_2022_latinamerica} devido a proximidade do ano de seu texto com a aprovação da Lei do Governo Digital em 2021.}. Já o Portal de Dados Abertos continha mais de 14 mil conjuntos de dados em 2022\footnote{Em 29/08/2025, o Portal da Dados Abertos contém 15.151 conjunto de dados.} sobre diversos temas, acessíveis em formatos legível para seres humanos e máquinas. 

\subsection{Digital Government Index}

A figura \ref{fig:comparacao_dgi_brasil_mundo_2020_2023} mostra os valores do DGI e suas dimensões para o Brasil e a OCDE definidos nas 1ª e 2ª edições do DGI, realizadas em 2020 e 2023, respectivamente. 

\begin{figure}[H]
	\centering
	\caption{Comparação do DGI do Brasil e da média da OECD de 2020 e 2023}
	\includegraphics[width=1\linewidth]{figuras/comparacao_dgi_brasil_mundo_2020_2023}
	\label{fig:comparacao_dgi_brasil_mundo_2020_2023}
	\footnotesize{Fonte: elaboração própria baseada em \cite{dgi_2020} e \cite{dgi_2023}.}
\end{figure}

Nota-se como o Brasil na pontuação geral - o \textbf{Composite Score} - sempre esteve acima da média da própria OCDE. A diferença sempre foi pequena (0,018 e 0,014, respectivamente). No quesito proatividade, avaliado pela dimensão \textbf{Proactiveness}, o Brasil se igualava à OCDE na 1ª edição do DGI, porém esteve à frente na 2ª edição, tal como ocorreu com a dimensão \textbf{Government as a Platform}.

Finalmente, tal mudança no resultado da 1ª edição do DGI para a 2ª representou a mudança trazida pela Lei do Governo Digital no paradigma da gestão pública brasileira. Os resultados apresentados pelo Brasil na 2ª edição do DGI mostraram a preocupação do Governo Federal com as peculiaridades do ambiente jurídico do Brasil e como ele poderia ser melhorado de maneira mais enérgica pela institucionalização, implementação e manutenção contínua  de conceitos como "Governo Digital", "Governo como Plataforma", e.g.

\subsection{Relatórios da OCDE}

Para \cite{going_digital_2022}, o relatório \textbf{Assessing national digital strategies and their governance}\footnote{cf. \url{https://www.oecd.org/content/dam/oecd/en/publications/reports/2022/05/assessing-national-digital-strategies-and-their-governance_9f3087a1/baffceca-en.pdf}} da OCDE propões uma nova metodologia para avaliar a abrangência para dar apoio a governo em desenvolvimento, monitorando e avaliando suas estratégias digitais nacionais.

\begin{figure}[H]
	\centering
	\caption{Going Digital Integrated Policy Framework e suas dimensões}
	\includegraphics[width=1\linewidth]{figuras/going_digital_integrated_policy_framework}
	\label{fig:ocde_going_digital}
	\footnotesize{Fonte: \cite{going_digital_2022}.}
\end{figure}

As dimensões presentes na figura \ref{fig:ocde_going_digital} são explicados na tabela \ref{tab:dimensoes_going_digital}.

\begin{longtable}[c]{@{}ll@{}}
	\caption{Explicação das dimensões do Going Digital Integrated Policy Framework}
	\label{tab:dimensoes_going_digital}\\
	\toprule
	\textbf{Dimensões} & \textbf{Descrição} \\
	\midrule
	\endfirsthead
	\toprule
	\textbf{Dimensões} & \textbf{Descrição} \\
	\midrule
	\endhead
	%
	\bottomrule
	\endfoot
	%
	\multicolumn{2}{r}{\textbf{Continua na próxima página.}} \\*
	\endfoot
	%
	\endlastfoot
	%
	Access &
	\begin{tabular}[c]{@{}l@{}}Access to communications infrastructures, services and data\\ underpin digital transformation and become more critical as \\ more people and devices go online.\end{tabular} \\ \midrule
	Use &
	\begin{tabular}[c]{@{}l@{}}The power and potential of digital technologies and data for people, \\ firms and governments depends on their effective use\end{tabular} \\ \midrule
	Innovation &
	\begin{tabular}[c]{@{}l@{}}Innovation pushes out the frontier of what is possible in the digital \\ age, driving job creation, productivity and sustainable growth.\end{tabular} \\ \midrule
	Jobs &
	\begin{tabular}[c]{@{}l@{}}As labour markets evolve, we must ensure that digital transformation \\  leads to more and better jobs and to facilitate just transitions from \\ one job to the next.\end{tabular} \\ \midrule
	Society &
	\begin{tabular}[c]{@{}l@{}}Digital technologies affect society in complex and interrelated ways, \\ and all stakeholders must work together to balance benefits and risks\end{tabular} \\ \midrule
	Trust &
	\begin{tabular}[c]{@{}l@{}}Trust in digital environments is essential; without it, an important \\ source of economic and social progress will be left unexploited.\end{tabular} \\ \midrule
	Market openness &
	\begin{tabular}[c]{@{}l@{}}Digital technologies change the way firms compete, trade and invest; \\ market openness creates an enabling environment for digital\\  transformation to flourish.\end{tabular} \\* \bottomrule
	\footnotesize{Fonte: elaboração baseada em \cite{portal_going_digital}.}
\end{longtable}

Haja vista o tabela \ref{tab:dimensoes_going_digital}, a figura \ref{fig:going_digital_policy_landscape_brasil} mostra o panorama das políticas públicas analisados pelas dimensões do \textbf{Framework}. 

\begin{figure}[H]
	\centering
	\caption{Panorama das políticas públicas do Going Digital: Brasil}
	\includegraphics[width=1\linewidth]{figuras/going_digital_policy_landscape_brasil}
	\label{fig:going_digital_policy_landscape_brasil}
	\footnotesize{Fonte: \cite{going_digital_policy_landscape}.}
\end{figure}

As políticas públicas assumidas pelo Brasil relativas às dimensões \textbf{Access} e \textbf{Use} representam metade da quantidade de iniciativas. À medida que a Lei do Governo Digital tem melhorado a Administração Pública, digitalizando serviços públicos, simplificando procedimentos administrativos e garantindo a participação cidadã na melhoria da gestão pública, o fato de metade das iniciativas listas pelo Panorama mostra o compromisso do Governo Federal com a modernização da Administração Pública.

O relatório \textbf{Assessing national digital strategies and their governance} tem um conjunto de dados associado. Tentou-se usá-lo para a elaboração de gráficos de radar para cada dimensão para o entendimento dos dados sejam facilitado. No entanto, em razão de sua complexidade para o autor deste trabalho, desistiu-se da iniciativa de gerar os gráficos. Assim, optou-se pelo uso da figura \ref{fig:nds_comprehensiveness_2022} presente no Apêndice \ref{imagens_grandes},  contudo, legibilidade não seja um de suas qualidades.\footnote{Os dados do \textbf{Framework Going Digital} podem ser baixados na url \url{https://goingdigital.oecd.org/}, seguindo as instruções presentes no Apêndice \ref{como_baixar_dados_going_digital}.}

Da figura \ref{fig:nds_comprehensiveness_2022} extra-se o que o Brasil não atingiu pontuação máxima em nenhuma dimensão, porém não atingiu pontuação baixa em nenhuma dimensão, diferentemente de alguns países-membros da OCDE, tais como a Dinamarca e Itália com sua pontuação baixa nas dimensões \textbf{Access} e \textbf{Market Openness}, respectivamente. As pontuações alcançadas pelo Brasil foram satisfatórias.

Complementarmente ao relatório \textbf{Assessing national digital strategies and their governance}, existe o relatório \textbf{OECD Digital Economy Outlook}\footnote{cf. \url{https://www.oecd.org/content/dam/oecd/en/publications/reports/2024/11/oecd-digital-economy-outlook-2024-volume-2_9b2801fc/3adf705b-en.pdf}}. Para \cite{digital_economy_outlook_v2_2024}, ele é a principal publicação que analisa tendências no desenvolvimento tecnológico, políticas digitais e desempenho digital nos países da OCDE e economias parceiras.

A edição de 2024 foi dividida em dois volumes. Segundo \cite{digital_economy_outlook_v2_2024}, o volume 2 examina as prioridades digitais, políticas públicas e arranjos de governança. Então, o 2º volume considera as fundações da transformação digital, análise de tendências em acesso e conectividade e as habilidades necessárias para prosperar em uma era digital.

O relatório \textbf{OECD Digital Economy Outlook} compartilha das mesmas dimensões do \textbf{Going Digital}. Do conjunto de dados do referido relatório, fez-se dois gráficos de radar para as dimensões para o Brasil e os países-membro da OCDE, presentes na figura \ref{fig:digital_economic_outlook_2024_brasil_ocde}.

\begin{figure}[H]
	\centering
	\caption{Dimensões do Going Digital}
	\includegraphics[width=0.75\linewidth]{figuras/digital_economic_outlook_2024_brasil_ocde}
	\label{fig:digital_economic_outlook_2024_brasil_ocde}
	\\ \footnotesize{Fonte: elaboração própria baseada em \cite{digital_economy_outlook_v2_2024}.}
\end{figure}

Nota-se como o Brasil ficou à frente dos valores alcançados pelos países-membro da OCDE, com exceção das dimensões \textbf{Access} e \textbf{Innovation}, onde tanto o Brasil, quanto os países-membro da OCDE se igualaram. Diferente do demonstrado na figura \ref{fig:nds_comprehensiveness_2022}, em que o Brasil estava em desvantagem em relação aos países da OCDE, em 2024 a situação foi revertida.