\chapter{Governo digital no Poder Judiciário}

O Brasil pós-democrático foi implementado como uma república  federativa, em concordância com \cite{cf88}, formada pela união indissolúvel dos Estados e Municípios e do Distrito Federal, constitui-se em Estado Democrático de Direito, cujos Poderes são o Executivo, Legislativo e Judiciário.

Haja vista o foco deste trabalho é o Poder Judiciário, esse Poder é composto, segundo \cite{cf88}, no art. 92 da Constituição Federal:

\begin{itemize}
    \item Supremo Tribunal Federal.
    \item Conselho Nacional de Justiça.
    \item Superior Tribunal de Justiça.
    \item Tribunal Superior do Trabalho.
    \item Tribunais Regionais Federais e Juízes Federais.
    \item Tribunais e Juízes do Trabalho.
    \item Tribunais e Juízes Eleitorais.
    \item Tribunais e Juízes Militares.
    \item  Tribunais e Juízes dos Estados e do Distrito Federal e Territórios.
\end{itemize}

\cite{cf88} concedeu ao Poder Judiciário autonomia administrativa e financeira. A importância da autonomia para o Poder Judiciário se confirma ao analisar as figuras seguintes relativas ao controle judicial sob o Poder Executivo,  corrupção no Poder Judiciário, Estado de Direito e a corrupção política.

A figura \ref{fig:judicial-constraints-on-the-executive-index} mostra a situação mundial do controle judicial sobre o Poder Executivo em 2024.

\begin{figure}[H]
	\centering
	\caption{Índice de controle judicial sobre o Poder Executivo}
	\includegraphics[width=1\linewidth]{figuras/judicial-constraints-on-the-executive-index.png}
	\label{fig:judicial-constraints-on-the-executive-index}
	\footnotesize{Fonte: \cite{jus_constraints_on_gov}.}
\end{figure}

Nota-se como o Poder Judiciário tem pouquíssimo controle sobre o Poder Executivo na em uma quantidade grande de países. A figura detalha o referido controle judicial no Brasil desde 1822 até 2024.

\begin{figure}[H]
    \centering
    \caption{Índice de controle judicial sobre o Poder Executivo no Brasil (1822-2024)}
    \includegraphics[width=1\linewidth]{figuras/judicial-constraints-on-the-executive-index-brazil.png}
    \label{fig:judicial-constraints-on-the-executive-index-brazil}
    \footnotesize{Fonte: \cite{jus_constraints_on_gov}.}
\end{figure}

No tocante a figura \ref{fig:judicial-constraints-on-the-executive-index-brazil}, nota-se como a democracia melhorou os índices do Brasil. Em 2024, o Brasil quase atingiu o valor máximo - 0,96 - enquanto a média mundial foi 0,664. Apenas 31 países de 193 - equivalente a 16\% do total - alcançaram uma pontuação de, no mínimo, 0,9 de 1,0 até o máximo.

A figura \ref{fig:quartis_controle_jus_sobre_gov} contém o Gráfico da caixa: índice de controle judicial sobre o Poder Executivo.

\begin{figure}[H]
    \centering
    \caption{Gráfico da caixa: índice de controle judicial sobre o Poder Executivo}
    \includegraphics[width=1\linewidth]{figuras/quartis_controle_jus_sobre_gov.png}
    \label{fig:quartis_controle_jus_sobre_gov}
    \footnotesize{Fonte: elaboração própia baseada em \cite{jus_constraints_on_gov}.}
\end{figure}

A figura \ref{fig:quartis_controle_jus_sobre_gov}, que mostra a distribuição do índice de controle judicial sobre o Poder Executivo, ilustra que no ano de 2024 teve um valor mínimo de 0,003 e um máximo de 0,988. A média dos dados foi de 0,664. Além disso, 25\% dos valores ficaram abaixo de 0,282 (1º quartil), enquanto 75\% dos valores foram inferiores a 0,829 (3º quartil).

A figura \ref{fig:judicial-corruption-score} mostra os índices globais de corrupção judiciária.

\begin{figure}[H]
	\centering
	\caption{Pontuação de corrupção judicial no mundo em 2024}
	\includegraphics[width=1\linewidth]{figuras/judicial-corruption-score.png}
	\label{fig:judicial-corruption-score}
	\footnotesize{Fonte: \cite{judicial-corruption-score}.}
\end{figure}

Nota-se que uma quantidade grande de países tem alta corrupção judiciária. A figura \ref{fig:judicial-corruption-score} complementa a anterior ao especificar a situação histórica do Brasil (desde 1822 até 2024).

A figura \ref{fig:judicial-corruption-score-brazil} mostra como o Brasil melhorou o aspecto da corrupção judiciária. 

\begin{figure}[H]
    \centering
    \caption{Pontuação de corrupção judicial no Brasil (1822-2024)}
    \includegraphics[width=1\linewidth]{figuras/judicial-corruption-score-brazil.png}
    \label{fig:judicial-corruption-score-brazil}
    \footnotesize{Fonte: \cite{judicial-corruption-score}.}
\end{figure}

Durante décadas, o Brasil ficou na faixa -1, alcançando 0 até 0,88. A atual pontuação do Brasil não está entre as melhores, pois ainda há as pontuações 2 e 3. No entanto, como a média mundial foi 0,249, o Brasil está acima da média mundial, porém o país foi superado por 66 países, cujas pontuações superaram 0,88. A quantidade de países que atingiu, no mínimo, 1 foi 29,84\%; no tocante a pontuação 2, foi 12\%; e por fim, 3 foi 3\% e nenhum atingiu 4 (valor máximo).

De maneira complementar, a figura \ref{fig:quartis_corrupcao_judiciaria} contém o diagrama de caixa da pontuação de corrupção judicial.

\begin{figure}[H]
    \centering
    \caption{Gráfico da caixa: corrupção judiciária}
    \includegraphics[width=1\linewidth]{figuras/quartis_corrupcao_judiciaria.png}
    \label{fig:quartis_corrupcao_judiciaria}
    \footnotesize{Fonte: elaboração própria baseada em \cite{judicial-corruption-score}.}
\end{figure}

A figura \ref{fig:quartis_corrupcao_judiciaria}, que mostra a distribuição do índice de corrupção judiciária, ilustra que no ano de 2024 teve um valor mínimo de -3,2610 e um máximo de 3,7690. A média dos dados foi de 0,2490. Além disso, 25\% dos valores ficaram abaixo de -0,9930 (1º quartil), enquanto 75\% dos valores foram inferiores a 1,4035 (3º quartil).

A figura \ref{fig:rule-of-law-index} mostra como a situação em 2024 do Estado de Direito no mundo.

\begin{figure}[H]
	\centering
	\caption{Estado de Direito no mundo em 2024}
	\includegraphics[width=1\linewidth]{figuras/rule-of-law-index.png}
	\label{fig:rule-of-law-index}
	\footnotesize{Fonte: \cite{rule-of-law-index}.}
\end{figure}

Nota-se como a instituição do Estado de Direito é muito fraco ou inexistente em muitos países do globo. De forma complementar a figura anterior, a figura \ref{fig:rule-of-law-index-brazil} mostra à situação histórica do Estado de Direito no Brasil (desde 1822 até 2024)

\begin{figure}[H]
	\centering
	\caption{Estado de Direito no Brasil (1822-2024)}
	\includegraphics[width=1\linewidth]{figuras/rule-of-law-index-brazil.png}
	\label{fig:rule-of-law-index-brazil}
	\footnotesize{Fonte: \cite{rule-of-law-index}.}
\end{figure}

Nota-se como o Brasil evoluiu no aspecto do Estado de Direito. Durante mais de 100 anos (1822 até quase 1950), o Brasil ficou na baixa faixa 0,2-0,4, no entanto, atingiu o valor de 0,8 em 2023 e 0,84 em 2024. O fato do Brasil não ter estado muitas décadas nas faixas 0,4-0,6 e 0,6-0,8 indica uma melhora significativa em seu índice de Estado de Direito, devido a políticas públicas que almejaram esse crescimento.

%\begin{figure}[H]
%	\centering
%	\caption{Gráfico da caixa: Estado de Direito}
%	\includegraphics[width=1\linewidth]{figuras/.png}
%	\label{fig:}
%	\footnotesize{Fonte: elaboração própria baseada em \cite{}.}
%\end{figure}

A figura \ref{}, que mostra a distribuição do índice de corrupção judiciária, ilustra que no ano de 2024 teve um valor mínimo de ???? e um máximo de ????. A média dos dados foi de ????. Além disso, 25\% dos valores ficaram abaixo de ???? (1º quartil), enquanto 75\% dos valores foram inferiores a ???? (3º quartil).

Além da análise gráfica, estudou-se se há correlação entre o índice de democracia eleitoral com controle judicial sob o Poder Executivo,  corrupção no Poder Judiciário e o Estado de Direito.

??????????

O Brasil está em posição privilegiada, pois, se apenas 25\% dos 193 países alcançaram uma pontuação superior e metade alcançou menos que a média mundial, isso é um indicativo de que autocracias são prevalentes. \cite{nord2025democracy} informa que o Brasil, juntamente com o Equador, Lesoto e a Polônia, pararam e reverteram processos de autocratização antes da disrupção da democracia, exibindo resiliência a rupturas autocráticas.

Outros dados que reforçam a importância da democracia no Brasil foram apresentados por \cite{nord2025democracy} no relatório \textbf{Democracy Report 2025} da V-Dem relativo ao ano de 2024 na lista abaixo:

\begin{itemize}
    \item Democracias liberais representam menos de 12\% da população mundial, ou seja, menos de 900 milhões de pessoas.
    \item Democracias eleitorais representam 17\% da população mundial.
    \item 40\%  da população mundial - 3,1 bilhões de pessoas - vive em países que estão em processo de autocratização.
    \item Há mais autocracias do que democracias no mundo: 91 contra 88. Em 2023, era o contrário.
    \item O mundo tem apenas 29 liberais, o que torna o regime o mesmo comum.
    \item 72\% das pessoas no mundo vivem em autocracias, percentual mais alto desde 1978.
\end{itemize}

Complementarmente, \cite{rule-of-law-index} mostra na figura \ref{fig:key-features-of-liberal-democracy} como os índices de uma democracia liberal no Brasil são altos.

\begin{figure}[H]
	\centering
	\caption{Característica principal de democracia eleitoral no Brasil em 2024}
	\includegraphics[width=1\linewidth]{figuras/key-features-of-liberal-democracy.png}
	\label{fig:key-features-of-liberal-democracy}
	\footnotesize{Fonte: \cite{rule-of-law-index}.}
\end{figure}

Como expresso pela figura \ref{fig:key-features-of-liberal-democracy}, além do controle judicial sobre o Poder Executivo e o Estado de Direito já citados em parágrafos anteriores, o controle legislativo sobre o Poder Executivo, as liberdades civis, as eleições democráticas são altas.

No tocante à liberdade de expressão e deassociação, bem como, os direitos humanos, a figuras \ref{}

As informações alarmantes apresentadas por \cite{nord2025democracy} expõem o quão benéfica tem sido a democracia para o Brasil desde a redemocratização. Embora o Brasil ainda apresente desafios, o país está em processo de melhoria institucional. Um exemplo disso é o fato do Brasil ter tido a capacidade de reverter uma tentativa de autocratização enquanto 40\%  da população mundial vive em países que estão em processo de autocratização.

Como consequência da argumentação anterior, \cite{pires2021paradoxo} corrobora a independência do Poder Judiciário. Para o autor, o Poder Judiciário obteve níveis elevados de independência com a Constituição Federal de 1988, que em um esforço para fortalecer a independência individual dos juízes, os termos e condições de mandato foram significativamente aprimorados.  Bem como, a Constituição Federal também fortaleceu a independência funcional do judiciário como instituição de governança, isolando-o do sistema político mais amplo.

Como resultado da independência proporcionada pela Constituição Federal, \cite{pires2021paradoxo} cita que os tribunais receberam controle total sobre seus assuntos administrativos, pessoais e disciplinares, de modo que o Poder Judiciário obteve controle quase total sobre seu orçamento.

Como forma de evitar a ingerência do Poder Executivo, \cite{pires2021paradoxo} argumenta que a Constituição Federal estabeleceu o STF é o responsável pela elaboração do orçamento anual da Justiça Federal e pelo encaminhamento direto ao Congresso Nacional. Assim, limitou-se o poder do Governo Federal sob o Poder Judiciário. 

Outro autor destacou a importância da independência do Poder Judiciário foi \cite{akutsu2012dimensoes}. Para ele, a importância de um Poder Judiciário independente dos Poderes Executivo e Legislativo decorre da necessidade de salvaguarda da liberdade individual dos cidadãos, que podem recorrer ao Judiciário contra abusos de autoridades de quaisquer dos três poderes. 

Para \cite{akutsu2012dimensoes}, a independência do Poder Judiciário não deve constituir óbice do cumprimento dos princípios e às normas da Constituição Federal pelos juízes. Além de poder serem responsabilizados perante os cidadãos. 

Complementarmente, para \cite{akutsu2012dimensoes} no caso da premissa da independência dos juízes e tribunais não se concretize, o desempenho do Poder Judiciário pode ser afetado, uma vez que os juízes enfrentariam óbices para  proferir sentenças que desagradassem pessoas afetadas por suas decisões.

Como fortalecedor da independência do Poder Judiciário, e como demonstração constitucional de sua importância para o Brasil, em 2004 o Congresso Nacional aprovou a Emenda à Constituição nº 45, de 2004. \cite{ec45_2004} criou o CNJ, as Súmulas Vinculantes do STF, extinguiu os Tribunais de Alçada e determinou sua incorporação aos Tribunais de Justiça, além das outras medidas estabelecidas.

No contexto da mudança legislativa promovida pela  Emenda à Constituição nº 45, de 2004, a criação do Conselho Nacional de Justiça (CNJ), através da publicação da referida Emenda à Constituição, foi precedida e sucedida de diversas celeumas relaciona das à sua natureza, constitucionalidade, legitimidade e efetividade.

Para \cite{silva2013transparencia}, a criação do CNJ, promovida pela publicação da Emenda à Constituição nº 45, de 2004, foi precedida e sucedida de diversas celeumas relaciona das à sua natureza, constitucionalidade, legitimidade e efetividade.  

Ao CNJ, segundo \cite{silva2013transparencia}, foram concedidos importantes poderes para que o órgão, respondendo do constituinte derivado, respondendo pelo controle da atuação administrativa e financeira do Poder Judiciário. São competências do CNJ, conforme \cite{cf88}, no art. 103-B, §4º, \textit{ipsi litteris}:

\noindent
\begin{flushleft}
	\setlength{\leftskip}{4cm}
	\small
	“§ 4º Compete ao Conselho o controle da atuação administrativa e financeira do Poder Judiciário e do cumprimento dos deveres funcionais dos juízes, cabendo-lhe, além de outras atribuições que lhe forem conferidas pelo Estatuto da Magistratura:
	
	I - zelar pela autonomia do Poder Judiciário e pelo cumprimento do Estatuto da Magistratura, podendo expedir atos regulamentares, no âmbito de sua competência, ou recomendar providências;
	
	II - zelar pela observância do art. 37 e apreciar, de ofício ou mediante provocação, a legalidade dos atos administrativos praticados por membros ou órgãos do Poder Judiciário, podendo desconstituí-los, revê-los ou fixar prazo para que se adotem as providências necessárias ao exato cumprimento da lei, sem prejuízo da competência do Tribunal de Contas da União;
	
	III - receber e conhecer das reclamações contra membros ou órgãos do Poder Judiciário, inclusive contra seus serviços auxiliares, serventias e órgãos prestadores de serviços notariais e de registro que atuem por delegação do poder público ou oficializados, sem prejuízo da competência disciplinar e correicional dos tribunais, podendo avocar processos disciplinares em curso, determinar a remoção ou à disponibilidade e aplicar outras sanções administrativas, assegurada ampla defesa;
	
	IV - representar ao Ministério Público, no caso de crime contra a administração pública ou de abuso de autoridade;
	
	V - rever, de ofício ou mediante provocação, os processos disciplinares de juízes e membros de tribunais julgados há menos de um ano;
	
	VI - elaborar semestralmente relatório estatístico sobre processos e sentenças prolatadas, por unidade da Federação, nos diferentes órgãos do Poder Judiciário;
	
	VII - elaborar relatório anual, propondo as providências que julgar necessárias, sobre a situação do Poder Judiciário no País e as atividades do Conselho, o qual deve integrar mensagem do Presidente do Supremo Tribunal Federal a ser remetida ao Congresso Nacional, por ocasião da abertura da sessão legislativa.”
\end{flushleft}         

As competências presentes no art. 103-B, §4º da Constituição Federal, atribuidas pelo Congresso Nacional, empoderam o CNJ como ordenador do Poder Judiciário . E como tal, o órgão tem modernizado o Poder Judiciário via seus atos administrativos. Consequentemente, serão analisados as políticas públicas de governo digital do Poder Judiciário Brasileiro aprovadas pelo CNJ, que modernizaram o poder judicante.

%Modernização do poder judiciário

%\section{Políticas públicas de governo digital no Judiciário}

%Creta

%Processo judicial eletrônico

%\Plataforma Digital do Poder Judiciário Brasileiro

%Justiça 4.0

%Justiça Aberta}

%DataJud

%https://www.cnj.jus.br/pesquisas-judiciarias/paineis-cnj/