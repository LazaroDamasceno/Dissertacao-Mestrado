\chapter{Referencial Teórico}

Para gerar os gráficos, fazer análises, determinar o valor e o método de correlação, fazer as tabelas e os mapas coropléticos, escolheu-se a linguagem de programação \textbf{Python}.

Foram usadas as bibliotecas \textbf{Pandas} para a análise exploratória de dados, \textbf{GeoPandas} para os mapas coropléticos, \textbf{Plotly} para a geração  de gráficos. Para a geração dos mapas foram usados os arquivos do projeto \textbf{Natural Earth} para os mapas do mundo e da malha municipal do Instituto Brasileiro de Geografia e Estatística. O tamanho das padronizado das figuras foi 10x6 polegadas, podendo ser utilizado outro, caso necessário.

Além dos argumentos anteriores, o coeficiente de correlação escolhido para todas as análises foi o de \textbf{Spearman}. \cite{hauke2011comparison} explica que o \textbf{coeficiente de correlação de Spearman} é uma estatística de postos não paramétrica (sem distribuição) proposta como uma medida da força da associação entre duas variáveis. É uma medida de uma associação monótona, usada quando a distribuição de dados torna o coeficiente de correlação de Pearson indesejável ou enganoso. 

Além disso, \cite{hauke2011comparison} esclarece que o \textbf{coeficiente de correlação de Spearman} não é uma medida da relação linear entre duas variáveis. Ele avalia quão bem uma função monotônica arbitrária pode descrever a relação entre duas variáveis, sem fazer quaisquer suposições sobre a distribuição de frequência das variáveis.

Ao contrário do \textbf{coeficiente de correlação de Pearson}, segundo \cite{hauke2011comparison}, ele não requer a suposição de que a relação entre as variáveis seja linear, nem requer que as variáveis sejam medidas em escalas intervalares; pode ser usado para variáveis medidas no nível ordinal.

Como forma de poder julgar qualquer valor de coeficiente de correlação encontrado, adotou-se a ideia de \cite{ali2022spearman}, presente na tabela \ref{tab:faixas-coeficiente-correlacao}.

\begin{longtable}[c]{@{}ll@{}}
	\caption{Faixas de coeficiente de correlação e seus significados}
	\label{tab:faixas-coeficiente-correlacao}\\
	\toprule
	%
	\endfirsthead
	%
	\toprule
	%
	\endhead
	%
	\textbf{Faixa} & \textbf{Significado} \\ \midrule
	1           & Correlação positiva completa. \\ \midrule
	0,99 - 0,70 & Correlação positiva forte.    \\ \midrule
	0,5 - 0,69  & Correlação positiva média.    \\ \midrule
	0,1 - 0,49  & Correlação positiva fraca.    \\ \midrule
	0           & Sem relação positiva.         \\ \bottomrule
	\footnotesize{Fonte: elaboração baseada em \cite{ali2022spearman}.}
\end{longtable}

Para qualquer coeficiente de correlação, adotar-se-ão os critérios presentes na tabela \ref{tab:faixas-coeficiente-correlacao} para determinar a existência ou não de correlação entre as variáveis comparadas.

Para as análises de regressão, optou-se pela polinomial ao invés da tradicional regressão linear. Similar a comparação com os coeficientes de correlação entre \textbf{Pearson} e \textbf{Spearman}, como a abordagem da análise dos coeficientes de correlação descartou relações lineares entre as variáveis analisadas, a regressão polinomial se alinha a escolha do coeficiente de correlação.

Finalmente, as figuras usadas neste trabalho, os códigos dos notebooks \textbf{Jupyter} usados para gerar as figuras e as bases da dados estão acessíveis em \url{https://github.com/LazaroDamasceno/Dissertacao-Mestrado-PoderJud-EGDI}.