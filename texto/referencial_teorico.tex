\chapter{Referencial Teórico}

Para gerar os gráficos, fazer análises, determinar o valor e o método de correlação, fazer as tabelas e os mapas coropléticos, escolheu-se a linguagem de programação \textbf{Python}.

Foram usadas as bibliotecas \textbf{Pandas} para a análise exploratória de dados, \textbf{GeoPandas} para os mapa coropléticos, \textbf{Matplotlib} e \textbf{Seaborn} para a geração  de gráficos.Usou-s a biblioteca \textbf{great\_tables} para gerar as tabelas em formato \textbf{PNG}. Para a geração dos mapas foram usados os arquivos do tipo \textbf{SHP} projeto \textbf{Natural Earth Data} para os mapa \textit{mundi} e da malha municipal do Instituto Brasileiro de Geografia e Estatística (IBGE). O tamanho das padronizado das figuras foi 10x6 polegadas, podendo ser utilizado outro, caso necessário.

Além dos argumentos anteriores, o coeficiente de correlação escolhido para todas as análises foi o de \textbf{Spearman}. \cite{hauke2011comparison} explica que o \textbf{coeficiente de correlação de Spearman} é uma estatística de postos não paramétrica (sem distribuição) proposta como uma medida da força da associação entre duas variáveis. É uma medida de uma associação monótona, usada quando a distribuição de dados torna o coeficiente de correlação de Pearson indesejável ou enganoso. 

Além disso, \cite{hauke2011comparison} esclarece que o \textbf{coeficiente de correlação de Spearman} não é uma medida da relação linear entre duas variáveis. Ele avalia quão bem uma função monotônica arbitrária pode descrever a relação entre duas variáveis, sem fazer quaisquer suposições sobre a distribuição de frequência das variáveis.

Ao contrário do \textbf{coeficiente de correlação de Pearson}, segundo \cite{hauke2011comparison}, ele não requer a suposição de que a relação entre as variáveis seja linear, nem requer que as variáveis sejam medidas em escalas intervalares; pode ser usado para variáveis medidas no nível ordinal.

Como forma de poder julgar qualquer valor de coeficiente de correlação encontrado, adotou-se a ideia de \cite{ali2022spearman}, presente na tabela \ref{tab:faixas-coeficiente-correlacao}.

\begin{table}[]
\caption{Faixas do coeficiente de correlação}
\begin{tabular}{@{}ll@{}}
\toprule
Valor      & Significado                  \\ \midrule
1          & Correlação positiva completa \\ \midrule
0.7 - 0.99 & Correlação positiva forte    \\ \midrule
0.5 - 0.69 & Correlação positiva média    \\ \midrule
0.1 - 0.49 & Correlação positiva fraca    \\ \midrule
0          & Sem relação positiva  \\ \midrule      
\label{tab:faixas-coeficiente-correlacao}
\footnotesize{Fonte: elaboração baseada em \cite{ali2022spearman}.}  
\end{tabular}
\end{table}

Para qualquer coeficiente de correlação presente neste trabalho, adotar-se-ão os critérios presentes na tabela \ref{tab:faixas-coeficiente-correlacao} para determinar a existência ou não de correlação entre as variáveis comparadas.