\chapter{Referencial Teórico}

Para gerar a maioria dos gráficos, fazer análises e determinar o valor e o método de correlação, utilizou-se a linguagem de programação \textbf{R}. Para a criação dos mapas coropléticos, foi usada a linguagem de programação \textbf{Python} em conjunto com as bibliotecas \textbf{Geopandas}, para a geração dos mapas e a leitura de arquivos geoespaciais (\textbf{GeoJSON}), e \textbf{Pandas}, para a manipulação dos dados tabulares provenientes de arquivos CSV, XLS e XLSX. Finalmente, a biblioteca \textbf{Matplotlib} foi empregada para salvar os mapas em formato PNG e para realizar ajustes na figura, garantindo que todos os seus elementos estivessem adequadamente dispostos.

O coeficiente de correlação escolhido para todas as análises foi o de Spearman, devido não se esperar linearidade das variáveis.