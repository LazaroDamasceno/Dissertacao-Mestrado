\chapter{Referencial Teórico}

Para gerar a maioria dos gráficos, fazer análises e determinar o valor e o método de correlação, utilizou-se a linguagem de programação \textbf{R}. Para a criação dos mapas coropléticos, foi usada a linguagem de programação \textbf{Python} em conjunto com as bibliotecas \textbf{Geopandas}, para a geração dos mapas e a leitura de arquivos geoespaciais (\textbf{GeoJSON}), e \textbf{Pandas}, para a manipulação dos dados tabulares provenientes de arquivos CSV, XLS e XLSX. Finalmente, a biblioteca \textbf{Matplotlib} foi empregada para salvar os mapas em formato PNG e para realizar ajustes na figura, garantindo que todos os seus elementos estivessem adequadamente dispostos.

O coeficiente de correlação escolhido para todas as análises foi o de Spearman. Diferentemente do coeficiente de Pearson, que pressupõe uma relação linear entre as variáveis, o coeficiente de Spearman avalia a relação monotônica, ou seja, a tendência de as variáveis se moverem juntas na mesma direção, seja de forma crescente ou decrescente, sem exigir que a relação seja estritamente linear.

A escolha por Spearman foi motivada pela observação de que alguns dos diagramas de dispersão, mesmo com a linha de regressão, não apresentaram uma relação completamente linear. A presença de pontos extremos (outliers) também foi um fator determinante, pois esses pontos poderiam distorcer significativamente o valor de um coeficiente de correlação linear como o de Pearson, levando a conclusões equivocadas. 

O coeficiente de Spearman, ao trabalhar com a ordenação (ranking) dos dados em vez dos valores brutos, é mais robusto a essas condições, refletindo de forma mais precisa a força e a direção da relação entre as variáveis. O valor do coeficiente varia de -1 a 1, onde um valor próximo de 1 indica uma forte relação monotônica positiva, um valor próximo de -1 indica uma forte relação monotônica negativa, e um valor próximo de 0 indica a ausência de uma relação monotônica.
